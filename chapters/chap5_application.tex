\chapter{Développement de l'Application Web}
\label{chap:application}

Ce chapitre présente la conception et le développement de l'application web GasTrack, réalisée avec la bibliothèque React.JS.

\section{Modèle de données}
\label{sec:modele_donnee}

\begin{figure}[H]
    \centering
    \includegraphics[width=0.9\textwidth]{chapters/diagramme_classe.png}
    \caption{Diagramme de classe}
    \label{fig:diagramme_classe}
\end{figure}

Ce modèle permet de structurer les informations échangées entre le backend cloud, l’application web et l’utilisateur.

\subsection{Vue d’ensemble du modèle}

Le modèle de données repose sur quatre classes principales :
\textit{Mesure}, \textit{Bouteille}, \textit{Passerelle} et \textit{Utilisateur}.  
Chaque classe représente une entité clé du système et encapsule les données ainsi que les comportements associés.

\subsection{Description des classes}

\subsubsection{Classe \texttt{Bouteille}}

La classe \texttt{Bouteille} représente la bouteille de gaz surveillée par le dispositif.

\textbf{Rôle :}
Elle centralise les informations statiques et dynamiques relatives à la bouteille, notamment son type, sa capacité et son état de remplissage.

\textbf{Responsabilités principales :}
\begin{itemize}
    \item Stocker les caractéristiques de la bouteille (capacité, type de gaz)
    \item Associer les mesures de niveau effectuées
    \item Fournir l’état courant du niveau de gaz
\end{itemize}

\subsubsection{Classe \texttt{Mesure}}

La classe \texttt{Mesure} modélise une mesure de niveau de gaz réalisée par le capteur ultrasonore.

\textbf{Rôle :}
Elle permet d’enregistrer les résultats des mesures et d’assurer l’historisation des données.

\textbf{Responsabilités principales :}
\begin{itemize}
    \item Stocker la valeur mesurée (niveau de gaz)
    \item Enregistrer la date et l’heure de la mesure
    \item Fournir les données nécessaires aux calculs de consommation et de prédiction
\end{itemize}

Chaque objet \texttt{Mesure} est associé à une unique \texttt{Bouteille}.


\subsubsection{Classe \texttt{Utilisateur}}

La classe \texttt{Utilisateur} représente l’utilisateur final de l’application web.

\textbf{Rôle :}
Elle permet de personnaliser l’utilisation du système et de relier les données à un utilisateur donné.

\textbf{Responsabilités principales :}
\begin{itemize}
    \item Consulter le niveau de gaz en temps réel
    \item Accéder à l’historique de consommation
    \item Visualiser les prédictions de consommation
\end{itemize}

Un utilisateur peut être associé à une ou plusieurs bouteilles.

\subsection{Relations entre les classes}

Les relations entre les classes assurent la cohérence du modèle :

\begin{itemize}
    \item Une \texttt{Bouteille} est associée à plusieurs \texttt{Mesure} (relation un-à-plusieurs), ce qui permet l’historisation des niveaux de gaz.
    \item L’\texttt{Utilisateur} consulte les données de sa \texttt{Bouteille} qui sont mises à jour via la \texttt{Passerelle} connectée au cloud.
\end{itemize}


\section{Analyse Fonctionnelle de l'Application}
\label{sec:analyse_app}

\subsection{Cas d'utilisation}

\begin{figure}[H]
    \centering
    \includegraphics[width=0.9\textwidth]{chapters/usecase.png}
    \caption{Diagramme de cas d'utilisation}
    \label{fig:use_cases_app}
\end{figure}

\subsection{Synthèse des cas d’utilisation}

Le tableau suivant présente une synthèse des principaux cas d’utilisation de l’application web.


\begin{table}[H]
\centering
\caption{Cas d'utilisation de l'application web}
\label{tab:cas_utilisation_app}
\small
\renewcommand{\arraystretch}{1.3}
\begin{tabular}{|p{2.8cm}|p{2.5cm}|p{4cm}|p{3.5cm}|p{2.2cm}|}
\hline
\rowcolor{blue!20}
\textbf{Cas d'utilisation} & 
\textbf{Précon-ditions} & 
\textbf{Scénario nominal} & 
\textbf{Scénarios alternatifs} & 
\textbf{Post-conditions} \\
\hline

\textbf{UC1:} Consulter le niveau de gaz & 
Connexion Internet active & 
\begin{enumerate}[leftmargin=*, nosep]
    \item L'utilisateur accède au site web
    \item Authentification automatique
    \item Récupération des dernières données serveur
    \item Affichage en temps réel
\end{enumerate} & 
\textbf{A1:} Pas d'internet → Mode hors ligne (cache)\newline
\textbf{A2:} Serveur inaccessible → Message d'erreur & 
Niveau de gaz affiché avec précision \\
\hline

\textbf{UC2:} Consulter l'historique & 
Application connectée & 
\begin{enumerate}[leftmargin=*, nosep]
    \item Accès à l'onglet "Historique"
    \item Téléchargement des données archivées
    \item Affichage graphique (jour/semaine/mois)
\end{enumerate} & 
\textbf{A1:} Historique vide → Message "Aucune donnée" & 
Historique visualisé sous forme de courbes \\
\hline

\textbf{UC3:} Obtenir une prédiction & 
Données suffisantes sur le serveur & 
\begin{enumerate}[leftmargin=*, nosep]
    \item Requête de prédiction au backend
    \item Réception du résultat calculé
    \item Affichage de la prédiction
\end{enumerate} & 
\textbf{A1:} Données insuffisantes → "Besoin de plus de données"\newline
\textbf{A2:} Consommation irrégulière → Prédiction imprécise & 
Prédiction affichée avec niveau de confiance \\
\hline

\end{tabular}
\end{table}


\section{Parcours Utilisateur}
\label{sec:parcours_utilisateur}

Cette section décrit le parcours utilisateur au sein de l’application web \textit{GasTrack}.  
Le parcours est basé sur les diagrammes d’activités correspondant aux trois principaux cas d’utilisation : 
la consultation du niveau de gaz, la consultation de l’historique de consommation et l’affichage des prédictions de consommation.

\subsection{Principes généraux du parcours}

Quel que soit le cas d’utilisation, le parcours utilisateur repose sur une séquence commune d’actions :
\begin{itemize}
    \item l'identification sécurisée,
    \item la sélection de la bouteille dans le parc,
    \item l’accès à la fonctionnalité demandée.
\end{itemize}

Cette approche garantit une expérience utilisateur cohérente, intuitive et homogène à travers l’ensemble de l’application.

\subsection{Parcours : Consultation du niveau de gaz}

Le premier parcours permet à l’utilisateur de consulter en temps réel le niveau de gaz restant dans la bouteille.

\begin{enumerate}
    \item L’utilisateur se connecte à l'application web.
    \item L'écran d'accueil affiche la liste des bouteilles associées avec leur dernier niveau connu (remonté par la passerelle).
    \item L’utilisateur sélectionne une bouteille pour voir les détails.
\end{enumerate}

Ce parcours se termine lorsque le niveau de gaz est correctement affiché à l’utilisateur.

\subsection{Parcours : Consultation de l’historique de consommation}

Ce parcours permet à l’utilisateur d’analyser l’évolution de sa consommation de gaz dans le temps.

\begin{enumerate}
    \item Depuis l'écran de détail d'une bouteille, l'utilisateur clique sur l'onglet "Historique".
    \item L’application interroge l'API REST pour obtenir les données agrégées.
\end{enumerate}

L’historique est présenté sous forme chronologique afin de faciliter l’analyse des tendances de consommation.

\subsection{Parcours : Affichage de la prédiction de consommation}

Ce parcours vise à fournir à l’utilisateur une estimation de la durée restante avant épuisement du gaz.

\begin{enumerate}
    \item Le backend calcule périodiquement les prédictions basées sur les nouvelles données LoRa.
    \item L'application affiche cette estimation (ex: "Il vous reste environ 12 jours") directement sur le tableau de bord.
\end{enumerate}

Ce parcours permet d’anticiper le remplacement de la bouteille de gaz et d’améliorer la gestion énergétique.

\subsection{Synthèse du parcours utilisateur}

L'utilisation d'une architecture cloud simplifie grandement le parcours utilisateur en supprimant les étapes techniques de connexion locale et d'appairage à chaque utilisation. L'information est disponible immédiatement après connexion au site.

\begin{figure}[H]
    \centering
    \includegraphics[width=0.9\textwidth]{chapters/activite_historique.png}
    \caption{Diagramme d'activité "Consulter Historique"}
    \label{fig:diagramme_activite}
\end{figure}

\begin{figure}[H]
    \centering
    \includegraphics[width=0.9\textwidth]{chapters/activite_niveau_gaz.png}
    \caption{Diagramme d'activité "Consulter Niveau de Gaz}
    \label{fig:activité_niveau_gaz}
\end{figure}

\begin{figure}[H]
    \centering
    \includegraphics[width=0.9\textwidth]{chapters/activite_prediction.png}
    \caption{Diagramme d'activité "Afficher Prédiction"}
    \label{fig:activité_prevision}
\end{figure}

\section{Interfaces Utilisateur}
\label{sec:interfaces_ui}

\begin{figure}[H]
    \centering
    \includegraphics[width=0.5\textwidth]{appli/Capture d’écran du 2025-12-30 19-37-47.png}
    \caption{Page d'accueil}
    \label{fig:accueil}
\end{figure}

\begin{figure}[H]
    \centering
    \includegraphics[width=0.5\textwidth]{appli/Capture d’écran du 2025-12-30 19-40-01.png}
    \caption{Formulaire d'utilisation 1}
    \label{fig:formulaire_1}
\end{figure}

\begin{figure}[H]
    \centering
    \includegraphics[width=0.5\textwidth]{appli/Capture d’écran du 2025-12-30 19-40-13.png}
    \caption{Formulaire d'utilisation 2}
    \label{fig:formulaire_2}
\end{figure}

\begin{figure}[H]
    \centering
    \includegraphics[width=0.5\textwidth]{appli/Capture d’écran du 2025-12-30 19-40-24.png}
    \caption{Formulaire d'utilisation 3}
    \label{fig:formulaire_3}
\end{figure}

\begin{figure}[H]
    \centering
    \includegraphics[width=0.5\textwidth]{appli/Capture d’écran du 2025-12-30 19-49-56.png}
    \caption{Tableau de bord}
    \label{fig:dashboard}
\end{figure}

\begin{figure}[H]
    \centering
    \includegraphics[width=0.5\textwidth]{appli/Capture d’écran du 2025-12-30 19-50-10.png}
    \caption{Historique}
    \label{fig:historique}
\end{figure}

\begin{figure}[H]
    \centering
    \includegraphics[width=0.5\textwidth]{appli/Capture d’écran du 2025-12-30 19-50-21.png}
    \caption{Prédictions}
    \label{fig:predictions}
\end{figure}

\begin{figure}[H]
    \centering
    \includegraphics[width=0.5\textwidth]{appli/Capture d’écran du 2025-12-30 19-50-33.png}
    \caption{Paramètres}
    \label{fig:params}
\end{figure}
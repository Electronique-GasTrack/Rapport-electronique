\chapter{Conception et Réalisation des Appareils}
\label{chap:appareil}

Ce chapitre décrit en détail la conception matérielle et logicielle des deux dispositifs constituant le système de surveillance du niveau de gaz.

\section{Appareil 1 : L'Émetteur (Mesure)}
\label{sec:emetteur}

L'émetteur est placé sous la bouteille de gaz. Son rôle est de mesurer la distance entre le fond de la bouteille et la surface du gaz liquide, puis de transmettre cette information par LoRa.

\subsection{Composants matériels}

\begin{itemize}
    \item \textbf{Arduino Nano} : Choisi pour sa compacité.
    \item \textbf{Capteur DYP-L06} : Capteur ultrasonore industriel avec sortie UART. Contrairement au JSN-SR04T, il offre une meilleure stabilité pour les mesures de niveau de liquide.
    \item \textbf{Module LoRa (433MHz)} : Configuré pour envoyer les données à l'appareil récepteur.
    \item \textbf{Batterie plate} : Pour une installation discrète sous le réservoir.
\end{itemize}

\subsection{Implémentation logicielle}

Le code de l'émetteur gère la lecture série du capteur DYP-L06. Les données sont reçues sous forme de trames hexadécimales commençant par 0xFF.

\begin{lstlisting}[language=C++, caption=Structure des données transmises]
struct SensorData {
  int level;
  unsigned long timestamp;
};
\end{lstlisting}

La distance est calculée en mm puis convertie en cm. Le niveau en pourcentage est obtenu par une fonction de transfert linéaire entre une distance "vide" (TANK\_HEIGHT) et une distance "plein" (FULL\_DISTANCE).

\section{Appareil 2 : Le Récepteur (Affichage et Alerte)}
\label{sec:recepteur}

Le récepteur est l'unité centrale de l'utilisateur, placée dans la cuisine ou le salon.

\subsection{Composants matériels}

\begin{itemize}
    \item \textbf{Arduino Nano} : Pilotage central.
    \item \textbf{Module LoRa (433MHz)} : Réception des paquets de données.
    \item \textbf{Écran LCD 16x2 I2C} : Affichage de la distance et du pourcentage.
    \item \textbf{LED RGB} : Signalisation lumineuse d'état (Verte, Orange, Rouge).
    \item \textbf{Buzzer actif} : Alerte sonore en cas de seuil critique.
\end{itemize}

\subsection{Logique d'affichage et d'alerte}

Le récepteur interprète la structure \texttt{SensorData} reçue et applique les règles suivantes :

\begin{itemize}
    \item \textbf{Niveau > 50\%} : LED Verte, pas d'alarme.
    \item \textbf{20\% < Niveau < 50\%} : LED Orange (R=255, G=165, B=0), pas d'alarme.
    \item \textbf{Niveau < 20\%} : LED Rouge et déclenchement du Buzzer actif.
\end{itemize}

L'utilisateur dispose de deux boutons : un pour couper l'alarme sonore (\texttt{BUTTON\_STOP\_PIN}) et un pour allumer/éteindre le rétroéclairage de l'écran (\texttt{BUTTON\_SCREEN\_PIN}).

\section{Algorithmes de mesure et transmission}

\subsection{Algorithme de l'émetteur}

L'émetteur fonctionne en boucle infinie avec un délai de 2 secondes. Il lit les données UART du DYP-L06, vérifie le checksum de la trame, calcule le pourcentage et envoie le tout via \texttt{LoRa.beginPacket()}.

\subsection{Algorithme du récepteur}

Le récepteur vérifie si un paquet LoRa est disponible via \texttt{LoRa.parsePacket()}. Si la taille correspond à la structure attendue, il met à jour ses variables globales et rafraîchit l'affichage LCD. Une sécurité (\texttt{TIMEOUT}) est implémentée : si aucun signal n'est reçu pendant 10 secondes, l'écran affiche "Pas de signal !" et la LED passe au bleu.

\section{Intégration Mécanique}
\label{sec:integration_mecanique}

\subsection{Installation sous la bouteille}

L'émetteur doit être solidement fixé sous la bouteille à l'aide d'un support adapté, en veillant à l'alignement vertical du capteur DYP-L06 pour garantir la réflexion de l'écho sur la surface du gaz liquide.

\subsection{Boîtier du récepteur}

Le récepteur est logé dans un boîtier compact permettant la visibilité de l'écran LCD et de la LED RGB, avec un accès facile aux boutons de commande.
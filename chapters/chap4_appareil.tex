\chapter{Conception et Réalisation de l'Appareil}
\label{chap:appareil}

Ce chapitre décrit en détail la conception matérielle et logicielle du dispositif embarqué de détection de gaz.

\section{Module de Détection}
\label{sec:module_detection}

\subsection{Principe physique de la détection ultrasonore}

\subsubsection{Propagation des ondes ultrasonores}

[INSÉRER EXPLICATION DÉTAILLÉE + FORMULES]

\begin{figure}[H]
    \centering
    \includegraphics[width=0.8\textwidth]{capteur_sonore.jpeg}
    \caption{Capteur Ultrasonore}
    \label{fig:capteur_ultrasons}
\end{figure}

\subsection{Mise en œuvre des capteurs JSN-SR04T}

\subsubsection{Caractéristiques techniques}

\begin{table}[H]
    \centering
    \caption{Spécifications capteur JSN-SR04T}
    \label{tab:specs_jsn}
    \begin{tabular}{ll}
        \toprule
        \textbf{Paramètre} & \textbf{Valeur} \\
        \midrule
        Tension d'alimentation & 5V DC \\
        Courant de travail & < 15 mA \\
        Fréquence ultrasonore & 40 kHz \\
        Portée de détection & 25 - 450 cm \\
        Précision & ±2 mm \\
        Angle de mesure & 75° \\
        Temps de cycle min & 60 ms \\
        Étanchéité & IP67 \\
        \bottomrule
    \end{tabular}
\end{table}

\subsubsection{Montage et fixation}

[DÉCRIRE SYSTÈME DE FIXATION + PHOTOS]

\subsection{Circuit de conditionnement du signal}

[SCHÉMA ÉLECTRIQUE DÉTAILLÉ + EXPLICATIONS]

\begin{figure}[H]
    \centering
    \fbox{\textcolor{blue}{[INSÉRER SCHÉMA : Circuit de conditionnement avec ampli-op LM358]}}
    \caption{Circuit de conditionnement}
    \label{fig:circuit_conditioning}
\end{figure}

\section{Module de Traitement}
\label{sec:module_traitement}

\subsection{Circuit Arduino Uno}

\subsubsection{Schéma de connexion complet}

\begin{figure}[H]
    \centering
    \fbox{\textcolor{blue}{[INSÉRER SCHÉMA FRITZING : Connexions complètes Arduino]}}
    \caption{Schéma de câblage complet}
    \label{fig:schema_cablage}
\end{figure}

\subsection{Algorithme de mesure}

\subsubsection{Organigramme général}

\begin{figure}[H]
    \centering
    \fbox{\textcolor{blue}{[INSÉRER FLOWCHART : Algorithme principal]}}
    \caption{Organigramme algorithme de mesure}
    \label{fig:flowchart_mesure}
\end{figure}



\subsection{Gestion de la mémoire EEPROM}

[EXPLICATION STOCKAGE CALIBRATION + HISTORIQUE]

\section{Module d'Affichage}
\label{sec:module_affichage}

\subsection{Écran OLED SSD1306}

\subsubsection{Configuration I2C}

[DÉTAILS CONFIGURATION + CODE]

\subsubsection{Interface graphique}

\begin{figure}[H]
    \centering
    \fbox{\textcolor{blue}{[INSÉRER MOCKUPS : Différents états d'affichage OLED]}}
    \caption{États d'affichage OLED}
    \label{fig:etats_oled}
\end{figure}

\subsection{Indicateurs LED}

[DESCRIPTION FONCTIONNEMENT LEDs]

\subsection{Alarme sonore}

[DESCRIPTION PATTERNS SONORES]

\section{Module de Communication BLE}
\label{sec:module_ble}

\subsection{Configuration module HM-10}

[DÉTAILS CONFIGURATION]

\subsection{Implémentation services GATT}

[CODE + EXPLICATIONS]

\section{Module d'Alimentation}
\label{sec:module_alimentation}

\subsection{Circuit d'alimentation}

\begin{figure}[H]
    \centering
    \fbox{\textcolor{blue}{[INSÉRER SCHÉMA : Circuit alimentation complet]}}
    \caption{Circuit d'alimentation}
    \label{fig:circuit_alimentation}
\end{figure}

\subsection{Gestion de la charge}

[EXPLICATIONS TP4056]

\section{Intégration Mécanique}
\label{sec:integration_mecanique}

\subsection{Conception du boîtier}

[PLANS + PHOTOS]

\subsection{Assemblage final}

[PHOTOS ÉTAPES D'ASSEMBLAGE]

\section{Conclusion du Chapitre}

\chapter{Conclusion Générale}
\label{chap:conclusion}

\section{Synthèse du Travail Réalisé}

L’objectif principal de ce projet était de concevoir et de réaliser un système intelligent de surveillance du niveau de gaz domestique capable de fonctionner de manière fiable, autonome et sécurisée. Pour répondre à cette problématique réelle, nous avons développé une solution distribuée innovante composée de deux unités complémentaires : un module de mesure autonome fixé sous la bouteille et un module de supervision déporté destiné à l’utilisateur.

L’architecture retenue repose sur la technologie LoRa, choisie pour sa portée étendue, sa robustesse face aux obstacles et sa faible consommation énergétique. Le module capteur, piloté par un microcontrôleur Arduino Nano et associé au capteur ultrasonore DYP-L06, permet une mesure non intrusive du niveau de gaz à travers la paroi métallique. Les données sont ensuite transmises sans fil vers l’unité d’affichage qui assure une visualisation claire et immédiate des informations.

Cette approche modulaire offre une solution à la fois flexible, sécurisée et adaptée aux contraintes domestiques, tout en respectant les exigences techniques du cahier des charges.

\section{Atteinte des Objectifs}

Les objectifs fixés en début de projet ont été atteints de manière satisfaisante :

\begin{itemize}
    \item \textbf{Mesure non intrusive :} L’utilisation des ultrasons permet d’estimer le niveau de gaz sans modification physique de la bouteille, garantissant sécurité et conformité.
    
    \item \textbf{Communication longue portée :} L’intégration de la technologie LoRa a permis de dépasser les limites des communications courte portée et d’assurer une transmission fiable sur plusieurs centaines de mètres.
    
    \item \textbf{Supervision conviviale :} Le module d’affichage avec écran LCD fournit une interface simple et lisible permettant un suivi en temps réel.
    
    \item \textbf{Autonomie énergétique :} L’optimisation de la consommation et l’usage de modes veille assurent un fonctionnement prolongé compatible avec une utilisation quotidienne.
\end{itemize}

Ainsi, la solution développée répond aux exigences fonctionnelles tout en restant réaliste sur les plans technique, économique et opérationnel.

\section{Apports Personnels et Compétences Acquises}

La réalisation de ce projet a constitué une expérience particulièrement enrichissante. Sur le plan technique, elle a permis de consolider des compétences en électronique embarquée, acquisition de données capteurs, communication radio longue portée et intégration matérielle. La confrontation aux contraintes physiques réelles a également permis de mieux comprendre les enjeux de fiabilité, de précision et de robustesse des systèmes embarqués.

Sur le plan organisationnel, le travail en équipe a favorisé le développement de compétences en gestion de projet, répartition des tâches, coordination technique et résolution collaborative de problèmes. Cette démarche nous a sensibilisés aux exigences de rigueur, de validation progressive et de documentation propres aux projets d’ingénierie.

Ce projet représente ainsi une véritable mise en situation professionnelle, proche des conditions réelles de conception d’un système technologique complet.

\section{Ouverture}

Au-delà de ses résultats immédiats, ce travail constitue une base solide pour le développement futur de solutions IoT appliquées à la gestion domestique. Il démontre l’intérêt d’une approche distribuée combinant capteurs intelligents, communication longue portée et interfaces utilisateur adaptées.

Il illustre également le potentiel des technologies embarquées pour répondre à des besoins concrets du quotidien, en améliorant à la fois le confort, la sécurité et l’efficacité énergétique des utilisateurs.
\chapter{Limites et Perspectives d'Évolution}
\label{chap:perspectives}

Ce chapitre présente les limites actuelles du système ainsi que les améliorations envisageables afin de renforcer ses performances, sa robustesse et son potentiel d’évolution vers une solution industrielle.

\section{Limites Actuelles}

Malgré son fonctionnement satisfaisant, le prototype présente certaines limites liées au cadre académique et aux ressources disponibles.

\subsection{Limites matérielles}

\begin{itemize}
    \item L’utilisation d’un unique capteur ultrasonore peut entraîner des variations de précision lorsque la bouteille est inclinée ou soumise à des vibrations.
    \item La qualité du couplage acoustique entre le capteur et la paroi influence directement la fiabilité des mesures.
    \item L’autonomie énergétique reste dépendante des composants utilisés et pourrait être améliorée par une optimisation plus poussée du circuit d’alimentation.
    \item Le prototype ne dispose pas encore d’un boîtier industriel certifié assurant une protection optimale sur le long terme.
\end{itemize}

\subsection{Limites fonctionnelles}

\begin{itemize}
    \item Le système fonctionne actuellement en supervision locale uniquement, sans accès distant aux données.
    \item Les algorithmes de traitement restent simples et ne prennent pas encore en compte certains paramètres environnementaux susceptibles d’affecter la mesure.
\end{itemize}

\section{Améliorations Possibles}

Plusieurs pistes d’amélioration peuvent être envisagées afin d’optimiser les performances globales du système.

\subsection{Améliorations matérielles}

\begin{itemize}
    \item Intégration de capteurs supplémentaires pour améliorer la précision et introduire une redondance de mesure.
    \item Utilisation d’un microcontrôleur plus performant et basse consommation afin d’augmenter l’autonomie.
    \item Conception d’un boîtier robuste, compact et étanche conforme aux normes industrielles.
    \item Optimisation du circuit d’alimentation et réduction des pertes énergétiques.
\end{itemize}

\subsection{Améliorations logicielles}

\begin{itemize}
    \item Développement d’une interface web ou mobile permettant la consultation à distance.
    \item Mise en place d’un système d’historisation des données pour l’analyse de consommation.
    \item Amélioration des algorithmes de traitement pour une estimation plus précise du niveau.
    \item Intégration de notifications intelligentes et d’alertes personnalisées.
\end{itemize}

\section{Perspectives d’Évolution}

À moyen et long terme, le système pourrait évoluer vers une solution complète et industrialisable. Les étapes envisagées incluent :

\begin{itemize}
    \item Validation du prototype en conditions réelles prolongées.
    \item Miniaturisation et optimisation du design électronique et mécanique.
    \item Certification selon les normes de sécurité et de compatibilité électromagnétique.
    \item Déploiement d’une infrastructure réseau permettant une supervision multi-utilisateurs.
\end{itemize}

À terme, cette solution pourrait être intégrée dans des environnements domotiques ou des plateformes IoT globales afin de participer à une gestion intelligente de l’énergie domestique. Elle pourrait également être adaptée à des contextes professionnels tels que la restauration, l’hôtellerie ou l’industrie.

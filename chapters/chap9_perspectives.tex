\chapter{Limites et Perspectives d'Évolution}
\label{chap:perspectives}

Ce chapitre présente les principales limites du système développé dans le cadre de ce projet,
ainsi que les perspectives d’amélioration et d’évolution possibles.
Il met en évidence les axes de progression tant sur le plan matériel que logiciel, et ouvre la voie
à une éventuelle transformation du prototype en une solution commerciale.

\section{Limites Actuelles}

Malgré les résultats satisfaisants obtenus, le système présente certaines limites inhérentes
au cadre académique et aux ressources disponibles.

\subsection{Limites matérielles}

\begin{itemize}
    \item Le dispositif repose sur un seul capteur ultrasonore, ce qui peut limiter la précision
    dans certaines conditions (inclinaison de la bouteille, vibrations).
    \item La fixation du capteur dépend fortement de la qualité du couplage acoustique,
    pouvant influencer la fiabilité des mesures.
    \item L’autonomie, bien que satisfaisante (environ 30 jours), reste limitée par l’utilisation
    de composants peu optimisés énergétiquement.
    \item Le prototype n’intègre pas encore un boîtier industriel certifié pour un usage prolongé.
\end{itemize}

\subsection{Limites logicielles}

\begin{itemize}
    \item Les algorithmes de prédiction reposent sur des modèles simples et ne tiennent pas
    encore compte de tous les paramètres d’usage (habitudes, saisonnalité).
\end{itemize}

\section{Améliorations Matérielles}

Plusieurs améliorations matérielles peuvent être envisagées afin d’augmenter les performances
et la fiabilité du système.

\begin{itemize}
    \item Utilisation de plusieurs capteurs ultrasonores pour améliorer la précision et introduire
    une redondance des mesures.
    \item Remplacement de l’Arduino Uno par un microcontrôleur plus performant et économe
    en énergie (ESP32, STM32).
    \item Intégration d’un boîtier robuste et étanche, conforme aux normes industrielles.
    \item Ajout de capteurs complémentaires (température, inclinaison) pour corriger et affiner
    les mesures.
    \item Optimisation du circuit d’alimentation pour augmenter l’autonomie de la batterie.
\end{itemize}

\section{Améliorations Logicielles}

Des évolutions logicielles significatives peuvent également être envisagées.

\begin{itemize}
    \item \textbf{Développement d'une application web et mobile :} Création d'une interface connectée pour permettre la consultation des niveaux à distance via Internet.
    \item Mise en place d'une infrastructure backend pour l'historisation des données dans le cloud.
    \item Amélioration des algorithmes de prédiction à l’aide de techniques d’apprentissage
    automatique basées sur l’historique de consommation.
    \item Ajout de notifications intelligentes sur smartphone (alertes personnalisées, prévisions avancées).
\end{itemize}

\section{Évolution vers une Solution Commerciale}

À plus long terme, le projet peut évoluer vers une solution commercialisable.

\subsection{Étapes envisagées}

\begin{itemize}
    \item Validation du prototype à travers des tests intensifs en conditions réelles.
    \item Miniaturisation du dispositif et optimisation du design industriel.
    \item Certification du produit selon les normes de sécurité et de compatibilité électromagnétique.
    \item Déploiement de l'application web sur une infrastructure cloud robuste et scalable.
    \item Mise en place d’une infrastructure backend scalable.
\end{itemize}

\subsection{Vision à long terme}

À terme, la solution pourrait être étendue à :
\begin{itemize}
    \item La gestion intelligente de la consommation énergétique domestique.
    \item L’intégration dans des systèmes domotiques et des plateformes IoT.
    \item Une utilisation à l’échelle industrielle ou commerciale (restaurants, hôtels, entreprises).
\end{itemize}

Ces perspectives montrent que le projet constitue une base solide pour des développements futurs,
tant académiques que professionnels.

\chapter{Difficultés Rencontrées et Solutions}
\label{chap:difficultes}

Ce chapitre présente les principales difficultés rencontrées tout au long de la réalisation du projet, ainsi que les solutions mises en œuvre pour les surmonter.  
Il met également en évidence les leçons apprises, tant sur le plan technique qu’organisationnel.

\section{Difficultés Matérielles}

\subsection{Difficulté d’acquisition du matériel}

L’une des principales contraintes rencontrées concerne l’acquisition du matériel nécessaire à la réalisation du dispositif embarqué.  
En particulier, la commande du capteur ultrasonore DYP-L06 a connu un retard important, s’étalant sur plus d’un mois. Ce retard a fortement impacté le planning initial du projet.

\textbf{Causes identifiées :}
\begin{itemize}
    \item Disponibilité limitée du capteur sur le marché local.
    \item Délais de livraison prolongés pour les commandes en ligne.
    \item Contraintes logistiques liées à l’importation du matériel.
\end{itemize}

\textbf{Solutions mises en œuvre :}
\begin{itemize}
    \item Réorganisation du planning en priorisant les tâches logicielles.
    \item Étude théorique approfondie du capteur en attendant sa réception.
    \item Simulation et préparation du code embarqué sans matériel physique.
\end{itemize}

\subsection{Contraintes financières}

La mobilisation des fonds nécessaires à l’achat des composants électroniques a également constitué une difficulté majeure.

\textbf{Problèmes rencontrés :}
\begin{itemize}
    \item Budget limité pour un projet de groupe.
    \item Retard dans la contribution financière de certains membres.
\end{itemize}

\textbf{Solutions adoptées :}
\begin{itemize}
    \item Réduction des coûts en choisissant des composants alternatifs lorsque possible.
    \item Mutualisation des ressources entre les membres du groupe.
    \item Planification progressive des achats selon les priorités.
\end{itemize}

\subsection{Disponibilité des bouteilles de gaz pour les tests}

La réalisation des tests en conditions réelles nécessitait la disponibilité de bouteilles de gaz vides et pleines, ce qui n’a pas toujours été évident.

\textbf{Solutions :}
\begin{itemize}
    \item Collaboration avec des particuliers et des commerces locaux.
    \item Utilisation de bouteilles partiellement remplies pour certains tests intermédiaires.
\end{itemize}

\section{Difficultés Logicielles}

\subsection{Courbe d’apprentissage de Flutter}

Le développement de l’application mobile avec Flutter a représenté un défi pour certains membres de l’équipe, notamment ceux n’ayant pas d’expérience préalable avec ce framework.

\textbf{Difficultés rencontrées :}
\begin{itemize}
    \item Prise en main du langage Dart.
    \item Compréhension de la gestion de l’état et de l’architecture de l’application.
    \item Mise en œuvre de la communication Bluetooth Low Energy (BLE).
\end{itemize}

\textbf{Solutions apportées :}
\begin{itemize}
    \item Répartition des tâches selon le niveau de compétence.
    \item Auto-formation à travers des tutoriels et la documentation officielle.
    \item Sessions de travail collaboratif et partage de connaissances.
\end{itemize}

\subsection{Déploiement du backend}

Le déploiement du backend de l’application a également posé plusieurs difficultés, notamment lors de la mise en production.

\textbf{Problèmes rencontrés :}
\begin{itemize}
    \item Configuration du serveur et des variables d’environnement.
    \item Problèmes de compatibilité entre la base de données et le backend.
    \item Gestion des erreurs lors du déploiement sur une plateforme cloud.
\end{itemize}

\textbf{Solutions mises en place :}
\begin{itemize}
    \item Tests locaux approfondis avant le déploiement.
    \item Utilisation de journaux (logs) pour identifier les erreurs.
    \item Simplification progressive de l’architecture backend.
\end{itemize}

\section{Difficultés d’Intégration}

L’intégration entre le dispositif embarqué, l’application mobile et le backend a constitué l’une des phases les plus complexes du projet.

\textbf{Difficultés principales :}
\begin{itemize}
    \item Instabilité initiale de la connexion Bluetooth entre l’appareil et le téléphone.
    \item Problèmes de synchronisation des données.
    \item Différences de comportement selon les modèles de smartphones.
    \item Gestion des pertes de connexion et des reconnexions.
\end{itemize}

\textbf{Solutions adoptées :}
\begin{itemize}
    \item Multiplication des tests sur différents appareils mobiles.
    \item Mise en place de mécanismes de reconnexion automatique.
    \item Validation progressive de chaque maillon de la chaîne de communication.
\end{itemize}

\section{Leçons Apprises}

La réalisation de ce projet a permis à l’équipe d’acquérir de nombreuses compétences et enseignements.

\textbf{Principales leçons retenues :}
\begin{itemize}
    \item L’importance d’une bonne planification et d’une gestion réaliste des délais.
    \item La nécessité d’anticiper les problèmes d’approvisionnement matériel.
    \item La valeur du travail en équipe et du partage des connaissances.
    \item L’intérêt de tester progressivement chaque composant avant l’intégration globale.
    \item L’adaptabilité face aux imprévus techniques et organisationnels.
\end{itemize}

Ces expériences ont contribué à renforcer les compétences techniques et organisationnelles des membres de l’équipe et constituent un apport significatif pour leurs futurs projets académiques et professionnels.

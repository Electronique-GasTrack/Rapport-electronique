\documentclass[11pt,a4paper]{article}

\usepackage[utf8]{inputenc}
\usepackage[T1]{fontenc}
\usepackage[french]{babel}
\usepackage{geometry}
\usepackage{setspace}

\geometry{margin=2cm}
\setstretch{1.1}

\begin{document}

\begin{center}
    \textbf{Nom du projet :} \\
    \textit{GasTrack – Système intelligent de suivi et de prédiction du niveau de gaz domestique}
    \normalsize
\end{center}

\vspace{0.3cm}



\vspace{0.3cm}

\textbf{Membres du groupe :}
\begin{itemize}
    \item AHMED JALIL TADIDA DJIDERE 
    \item AKOULOUZE JAMALI AMINA WENDY
    \item AKOULOUZE MANY EVA FELICIA MEG
    \item ATCHINE OUDAM HANNIEL
    \item ESSONO SANDRINE FLORA
    \item FEZE DJOUMESSI FRED
    \item NBEUYO KOLOGNE ULRICH
    \item TAGATSING FOTSING SAMUEL SEAN
    \item TEKEU KAMCHI NATHAN
\end{itemize}

\vspace{0.2cm}

\section*{Objectifs du projet}

L’objectif principal de ce projet est de concevoir et réaliser un système intelligent permettant de surveiller le niveau de gaz dans une bouteille domestique et d’en prédire la consommation.  
Le système vise à améliorer la sécurité, le confort d’utilisation et l’anticipation du remplacement des bouteilles de gaz, tout en offrant une solution accessible et à faible consommation énergétique.

\vspace{0.2cm}

\section*{Description du cahier des charges}

Le système doit permettre :
\begin{itemize}
    \item la mesure fiable du niveau de gaz à l’aide de capteurs ultrasonores,
    \item la transmission sans fil des données vers une application mobile,
    \item l’affichage du niveau de gaz en temps réel,
    \item la consultation de l’historique de consommation,
    \item la prédiction de la durée restante d’utilisation,
    \item une faible consommation énergétique garantissant une autonomie d’au moins 30 jours.
\end{itemize}

Le dispositif doit être sécurisé, simple à utiliser et compatible avec les smartphones Android et iOS.

\vspace{0.2cm}

\section*{Paramètres et technologies utilisés}

\textbf{Matériel :}
\begin{itemize}
    \item Microcontrôleur : Arduino
    \item Capteurs : Capteur ultrasonore (DYP-L06)
    \item Communication : Bluetooth Low Energy (BLE)
    \item Affichage local : Écran OLED SSD1306
    \item Alimentation : Batterie rechargeable
\end{itemize}

\vspace{0.5cm}
\textbf{Logiciel :}
\begin{itemize}
    \item Application mobile : Flutter
    \item Communication : BLE
    \item Backend : Spring Boot
    \item Base de données : PostgreSQL
\end{itemize}

\end{document}

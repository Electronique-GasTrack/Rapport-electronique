\documentclass[12pt,a4paper]{report}

%=============================================================================
% PACKAGES
%=============================================================================
\usepackage[utf8]{inputenc}
\usepackage[T1]{fontenc}
\usepackage[french,provide=*]{babel}
\usepackage{geometry}
\geometry{
    a4paper,
    left=2cm,
    right=2cm,
    top=2cm,
    bottom=2cm,
    headheight=15pt
}

% Packages mathématiques et scientifiques
\usepackage{amsmath,amssymb,amsfonts}

% Packages graphiques
\usepackage{graphicx}
\usepackage{float}
\usepackage{caption}
\usepackage{subcaption}
\usepackage{booktabs}      % Pour de belles lignes dans les tableaux
\usepackage{tabularx}      % Pour des tableaux avec des colonnes à largeur ajustable
\usepackage[table]{xcolor} % Pour colorer les cellules des tableaux
\usepackage[table]{xcolor}
\usepackage{tikz}
\usetikzlibrary{shapes,arrows,positioning,calc}

% Packages pour les tableaux
\usepackage{booktabs}
\usepackage{multirow}
\usepackage{longtable}
\usepackage{array}

% Packages pour le code
\usepackage{listings}

% Packages divers
\usepackage{fancyhdr}
\usepackage{titlesec}
\usepackage{tocloft}
\usepackage{enumitem}
\usepackage{pdfpages}
\usepackage{csquotes} % Recommandé avec babel
\usepackage[most]{tcolorbox}


% Packages pour les liens et références (à charger en dernier)
\usepackage{hyperref}
\usepackage[capitalise,noabbrev]{cleveref}

%=============================================================================
% CONFIGURATION DES PACKAGES
%=============================================================================

% Configuration hyperref
\hypersetup{
    colorlinks=true,
    linkcolor=blue,
    filecolor=magenta,
    urlcolor=cyan,
    citecolor=green,
    pdftitle={Détecteur de Niveau de Gaz Butane},
    pdfauthor={Équipe Projet GIF-1},
    pdfsubject={Rapport de Projet},
    pdfkeywords={Arduino, Ultrason, IoT, Gaz, BLE},
    bookmarksnumbered=true,
}

% Configuration des listings pour le code
\lstset{
    basicstyle=\ttfamily\small,
    keywordstyle=\color{blue}\bfseries,
    commentstyle=\color{green!60!black}\itshape,
    stringstyle=\color{red},
    showstringspaces=false,
    breaklines=true,
    frame=single,
    numbers=left,
    numberstyle=\tiny\color{gray},
    captionpos=b,
}

% Configuration des en-têtes et pieds de page
\pagestyle{fancy}
\fancyhf{}
\fancyhead[L]{\leftmark}
\fancyfoot[C]{\thepage}  % Numéro de page en bas au centre
\renewcommand{\headrulewidth}{0.5pt}
\renewcommand{\footrulewidth}{0pt}

\fancypagestyle{plain}{
    \fancyhf{}
    \fancyfoot[C]{\thepage}  % Numéro de page en bas au centre
    \renewcommand{\headrulewidth}{0pt}
    \renewcommand{\footrulewidth}{0pt}
}

% Configuration des titres de chapitres
\titleformat{\chapter}[display]
{\normalfont\huge\bfseries\color{blue!70!black}}
{\chaptertitlename\ \thechapter}{20pt}{\Huge}

% Configuration captions
\captionsetup{
    labelfont=bf,
    justification=centering,
    singlelinecheck=true
}





% Personnalisation des listes
\setlist[itemize]{itemsep=2pt, topsep=5pt}
\setlist[enumerate]{itemsep=2pt, topsep=5pt}

%=============================================================================
% COMMANDES PERSONNALISÉES
%=============================================================================


% Commande pour les références croisées
\newcommand{\figref}[1]{Figure~\ref{#1}}
\newcommand{\tabref}[1]{Tableau~\ref{#1}}
\newcommand{\secref}[1]{Section~\ref{#1}}
\newcommand{\chapref}[1]{Chapitre~\ref{#1}}

% Couleurs personnalisées
\definecolor{uniblue}{RGB}{0,51,102}
\definecolor{warningred}{RGB}{204,0,0}
\definecolor{successgreen}{RGB}{0,153,51}

%=============================================================================
% DÉBUT DU DOCUMENT
%=============================================================================

\begin{document}

%-----------------------------------------------------------------------------
% PAGE DES PARTICIPANTS (non numérotée)
%-----------------------------------------------------------------------------
\pagenumbering{gobble} % Désactive la numérotation
\begin{titlepage}
    \centering
    % --- Titre de la page ---
    {\Huge\bfseries\color{blue!70!black} Participants au Projet\par}
    
    \vspace{0.25cm}
    
    % --- Début du tableau ---
    \begingroup
    \setlength{\tabcolsep}{10pt} % Espacement entre les colonnes
    \renewcommand{\arraystretch}{1.8} % Hauteur des lignes
    
    \begin{tabularx}{\textwidth}{|l|l|c|>{\raggedright\arraybackslash}X|}
        \hline
        \rowcolor{blue!70!black} % Couleur de fond pour l'en-tête
        \textcolor{white}{\textbf{Nom et Prénom}} & \textcolor{white}{\textbf{Matricule}} & \textcolor{white}{\textbf{\% Part.}} & \textcolor{white}{\textbf{Travail réalisé}} \\
        \hline
        
        AHMED JALIL TADIDA & 22P265 & 11.2\% & Recherche sur le module de détection et rédaction du rapport \\
        \hline
        AKOULOUZE JAMALI AMINA & 22P340 & 11.1\% & Recherche et implémentation du module d'affichage \\
        \hline
        AKOULOUZE MANY EVA  & 22P508 & 11.1\% & Implémentation de la simulation du projet \\
        \hline
        ATCHINE OUDAM HANNIEL & 22P590 & 11.1\% & Implémentation du module de détection et modélisation 3D\\
        \hline
        ESSONO SANDRINE FLORA & 22P289 & 11.1\% & Implémentation de la simulation sur Proteus \\
        \hline
        FEZE DJOUMESSI FRED & 22P334 & 11.1\% & Recherche du module d'affichage \\
        \hline
        NBEUYO KOLOGNE ULRICH & 22P204 & 11.1\% & Recherche et implémentation du module d'alimentation \\
        \hline
        TAGATSING FOTSING SAMUEL & 22P215 & 11.2\% & Recherche et implémentation de la simulation du projet et rédaction du rapport \\
        \hline
        TEKEU KAMCHI NATHAN & 22P245 & 11\% & Recherche sur le module d'alimentation\\
        \hline
    \end{tabularx}
    \endgroup
    % --- Fin du tableau ---
    
\end{titlepage}

% Commande pour s'assurer que la page suivante (table des matières) commence bien
\cleardoublepage 
\pagenumbering{arabic} % Réactive la numérotation en chiffres arabes

%-----------------------------------------------------------------------------
% TABLE DES MATIÈRES
%-----------------------------------------------------------------------------
\tableofcontents
\clearpage

%-----------------------------------------------------------------------------
% LISTE DES FIGURES
%-----------------------------------------------------------------------------
\phantomsection
\addcontentsline{toc}{chapter}{\listfigurename}
\listoffigures
\clearpage

%-----------------------------------------------------------------------------
% LISTE DES TABLEAUX
%-----------------------------------------------------------------------------
\phantomsection
\addcontentsline{toc}{chapter}{\listtablename}
\listoftables
\clearpage

%-----------------------------------------------------------------------------
% LISTE DES ACRONYMES ET ABRÉVIATIONS
%-----------------------------------------------------------------------------
\chapter*{Liste des Acronymes et Abréviations}
\addcontentsline{toc}{chapter}{Liste des Acronymes et Abréviations}

\begin{tabular}{ll}
    \textbf{ADC} & Analog-to-Digital Converter (Convertisseur Analogique-Numérique) \\
    \textbf{API} & Application Programming Interface \\
    \textbf{BLE} & Bluetooth Low Energy \\
    \textbf{FCFA} & Franc de la Communauté Financière Africaine \\
    \textbf{FP} & Fonction Principale \\
    \textbf{FS} & Fonction Secondaire \\
    \textbf{GPL} & Gaz de Pétrole Liquéfié \\
    \textbf{I2C} & Inter-Integrated Circuit \\
    \textbf{IDE} & Integrated Development Environment \\
    \textbf{IoT} & Internet of Things (Internet des Objets) \\
    \textbf{LoRa} & Long Range \\
    \textbf{LCD} & Liquid Crystal Display \\
    \textbf{LED} & Light-Emitting Diode \\
    \textbf{Li-ion} & Lithium-ion \\
    \textbf{Li-Po} & Lithium-Polymère \\
    \textbf{MCU} & Microcontroller Unit \\
    \textbf{OLED} & Organic Light-Emitting Diode \\
    \textbf{PCB} & Printed Circuit Board \\
    \textbf{PWM} & Pulse Width Modulation \\
    \textbf{REST} & Representational State Transfer \\
    \textbf{SRAM} & Static Random-Access Memory \\
    \textbf{SPI} & Serial Peripheral Interface \\
    \textbf{UART} & Universal Asynchronous Receiver-Transmitter \\
    \textbf{UI} & User Interface \\
    \textbf{UX} & User Experience \\
    \textbf{UUID} & Universally Unique Identifier \\
\end{tabular}

\clearpage

%=============================================================================
% CORPS DU RAPPORT
%=============================================================================

% Inclusion des chapitres (à créer dans des fichiers séparés)
\chapter{Introduction Générale}
\label{chap:introduction}

\section{Contexte}
\label{sec:contexte}

\subsection{Utilisation du gaz domestique}

L'utilisation du gaz butane comme source d'énergie domestique est largement répandue dans les ménages camerounais et dans de nombreux pays en développement. Ce gaz de pétrole liquéfié (GPL) représente une alternative économique et pratique au bois de chauffage et au charbon, contribuant ainsi à la réduction de la déforestation et des émissions de gaz à effet de serre.

Au Cameroun, les bouteilles de gaz domestique sont disponibles en plusieurs capacités (6 kg, 12,5 kg et 31 kg) et sont utilisées quotidiennement pour la préparation des repas dans une grande majorité des foyers urbains et périurbains. Selon les résultats de l’Enquête Intégrée sur les Conditions de Vie des Ménages (EICVM / MIS 2022) menée par l’Institut National de la Statistique (INS), environ 48,7 \% des ménages urbains utilisent le gaz butane (GPL) comme principale source d’énergie pour la cuisson.

\begin{figure}[H]
    \centering
    \includegraphics[width=0.4\textwidth]{chapters/bouteille_gaz.jpg}
    \caption{Bouteiile de gaz standard}
    \label{fig:Bouteille_gaz}
\end{figure}

\subsection{Problèmes réels rencontrés}

Malgré les avantages du gaz butane, son utilisation présente plusieurs inconvénients majeurs qui affectent le confort et la sécurité des utilisateurs :

\paragraph{Coupures imprévues :}
L'impossibilité de connaître avec précision le niveau de gaz restant dans la bouteille entraîne fréquemment des coupures imprévues, généralement au moment le moins opportun, notamment pendant la préparation des repas. Cette situation génère des désagréments importants et perturbe l'organisation domestique.

\paragraph{Manipulations dangereuses :}
Face à l'absence d'indicateur de niveau, les utilisateurs sont contraints de recourir à des méthodes empiriques pour estimer le contenu restant :
\begin{itemize}
    \item \textbf{Soulèvement répété de la bouteille} pour en estimer le poids, ce qui peut causer des blessures dorsales, particulièrement pour les personnes âgées ou à mobilité réduite ;
    \item \textbf{Secouement de la bouteille} pour tenter d'entendre le liquide, méthode imprécise et potentiellement dangereuse.
\end{itemize}

\paragraph{Risques de sécurité :}
Ces manipulations répétées augmentent considérablement les risques :
\begin{itemize}
    \item Détérioration des raccords et joints, pouvant entraîner des fuites de gaz ;
    \item Chutes de la bouteille lors des manipulations, avec risque d'explosion en cas de choc violent ;
    \item Exposition prolongée au gaz en cas de fuite non détectée.
\end{itemize}

\paragraph{Mauvaise planification :}
L'incertitude sur le niveau de gaz restant empêche une planification efficace des achats de recharge, conduisant soit à des remplacements prématurés (gaspillage économique), soit à des ruptures de stock au moment critique.

\section{Problématique}
\label{sec:problematique}

\subsection{Formulation du problème}

La question centrale qui motive ce projet peut être formulée ainsi :

\begin{quote}
    \textit{\textbf{Comment mesurer de manière fiable, sécurisée et non-intrusive le niveau de gaz restant dans une bouteille domestique, tout en fournissant à l'utilisateur une information claire et accessible en temps réel ?}}
\end{quote}

Cette problématique soulève plusieurs défis techniques et pratiques :

\begin{enumerate}
    \item \textbf{Défi de la mesure non-intrusive} : La solution doit fonctionner sans modifier la structure de la bouteille, sans percer ou ouvrir le récipient sous pression, et sans contact direct avec le gaz liquéfié.
    
    \item \textbf{Défi de la précision} : Le système doit fournir une mesure suffisamment précise pour permettre une planification efficace, tout en tenant compte des contraintes liées à la paroi métallique et aux variations de température.
    
    \item \textbf{Défi de la sécurité} : Toute solution proposée doit respecter des normes de sécurité strictes, notamment l'absence d'étincelles, l'utilisation de composants basse tension, et la robustesse face aux conditions d'utilisation domestique.
    
    \item \textbf{Défi de l'accessibilité} : Le système doit être abordable, facile à installer et à utiliser, sans nécessiter de compétences techniques particulières.
    
    \item \textbf{Défi de l'autonomie} : L'appareil doit fonctionner de manière autonome avec une alimentation par batterie, sans nécessiter de connexion permanente au secteur.
\end{enumerate}

\subsection{Limites des solutions existantes}

Plusieurs approches ont été proposées ou commercialisées pour résoudre ce problème, mais chacune présente des limitations significatives :

\begin{table}[H]
    \centering
    \caption{Comparaison des solutions existantes}
    \label{tab:solutions_existantes}
    \begin{tabular}{p{3cm}p{5cm}p{5cm}}
        \toprule
        \textbf{Solution} & \textbf{Avantages} & \textbf{Inconvénients} \\
        \midrule
        Pesage manuel & Simple, pas d'électronique & Nécessite manipulation, imprécis, pénible \\
        \midrule
        Balance connectée & Précise, automatique & Coût élevé, encombrant, nécessite repositionnement \\
        \midrule
        Capteur de pression & Mesure directe & Intrusif, risque de fuite, installation complexe \\
        \midrule
        Capteur thermique externe & Non-intrusif & Imprécision, sensible aux conditions ambiantes \\
        \midrule
        Jauges mécaniques & Simples & Peu fiables, lecture approximative \\
        \bottomrule
    \end{tabular}
\end{table}

\subsection{Justification de l'approche ultrasonore}

Face à ces limitations, nous avons opté pour une solution basée sur la \textbf{détection ultrasonore externe}, qui présente les avantages suivants :

\begin{itemize}
    \item \textbf{Non-intrusif} : Aucune modification de la bouteille n'est requise ;
    \item \textbf{Sûr} : Pas de contact avec le gaz, utilisation de composants basse tension ;
    \item \textbf{Précis} : Capable de détecter l'interface liquide-gaz à travers la paroi métallique ;
    \item \textbf{Économique} : Coût des composants raisonnable et accessible ;
    \item \textbf{Fiable} : Technologie éprouvée dans d'autres applications industrielles.
\end{itemize}

\section{Objectifs du Projet}
\label{sec:objectifs}

\subsection{Objectif général}

L'objectif principal de ce projet est de :

\begin{quote}
    \textit{Concevoir, réaliser et valider un système électronique intelligent permettant la mesure non-intrusive du niveau de gaz butane dans une bouteille domestique, avec transmission sans fil et affichage local déporté.}
\end{quote}

\subsection{Objectifs spécifiques}

Pour atteindre cet objectif général, nous avons défini les objectifs spécifiques suivants :

\subsubsection{OS1 : Mesure non-intrusive du niveau de gaz}

Développer un module de détection basé sur des capteurs ultrasonores capable de :
\begin{itemize}
    \item Mesurer la hauteur du liquide dans la bouteille sans contact direct avec le gaz ;
    \item Atteindre une précision de $\pm$5\% du volume total ;
    \item Fonctionner avec différentes tailles de bouteilles (6 kg, 12,5 kg, 13 kg) ;
    \item Réaliser une mesure complète en moins de 5 secondes.
\end{itemize}

\subsubsection{OS2 : Affichage et supervision déportés}

Concevoir un dispositif de supervision distinct permettant de :
\begin{itemize}
    \item Recevoir les données du capteur à distance via une liaison radio ;
    \item Visualiser le niveau de gaz sur un écran dédié (LCD/OLED) ;
    \item Émettre des alertes sonores et visuelles en cas de niveau critique ;
    \item Fonctionner de manière autonome dans une pièce de vie.
\end{itemize}


\subsubsection{OS3 : Communication LoRa}

Mettre en place une architecture de communication comprenant :
\begin{itemize}
    \item Une liaison radio longue portée (LoRa) entre le capteur et l'afficheur ;
    \item Un protocole de communication optimisé pour la faible consommation.
\end{itemize}

\subsubsection{OS4 : Sécurité et fiabilité}

Garantir la sécurité du système en :
\begin{itemize}
    \item Utilisant exclusivement des composants basse tension ($\leq$ 5V) ;
    \item Assurant l'isolation électrique totale entre circuits et bouteille ;
    \item Respectant les normes de sécurité électrique domestique ;
    \item Implémentant des protections contre les courts-circuits et surcharges ;
    \item Validant le système par des tests rigoureux.
\end{itemize}

\subsubsection{OS5 : Autonomie et efficacité énergétique}

Optimiser la consommation énergétique pour :
\begin{itemize}
    \item Atteindre une autonomie minimale de 30 jours avec batterie rechargeable ;
    \item Implémenter des modes de veille intelligents ;
    \item Permettre la recharge via port USB standard ;
    \item Afficher l'état de charge de la batterie.
\end{itemize}

\section{Méthodologie de Travail}
\label{sec:methodologie}

Pour mener à bien ce projet, nous avons adopté une approche structurée en plusieurs phases :

\subsection*{Phase 1 : Étude et conception (3 semaines)}

\subsection*{Phase 2 : Développement matériel (4 semaines)}

\subsection*{Phase 3 : Développement logiciel (4 semaines)}

\subsection*{Phase 4 : Tests et validation (2 semaines)}

\subsection*{Phase 5 : Documentation (2 semaines)}

\chapter{Analyse du Projet}
\label{chap:analyse}

Ce chapitre présente l'analyse approfondie des besoins, des contraintes et des spécifications fonctionnelles du système de détection de gaz. Cette analyse constitue le fondement sur lequel repose toute la conception du projet.

\section{Étude des Solutions Existantes}
\label{sec:solutions_existantes}

Cette section présente une analyse comparative des principales technologies existantes permettant d'estimer le niveau de gaz dans des récipients fermés, notamment les bouteilles de gaz domestique. Chaque méthode est étudiée selon son principe de fonctionnement, ainsi que ses avantages et ses limites dans le contexte du projet.

\subsection{Méthodes de mesure de niveau}

\subsubsection{Mesure par pesage}

\textbf{Principe :}  
La méthode par pesage consiste à déterminer la quantité de gaz restante en mesurant le poids total de la bouteille. En connaissant le poids de la bouteille vide, il est possible de déduire la masse de gaz restante par simple soustraction. Cette approche repose sur le fait que la masse du gaz diminue progressivement au fur et à mesure de son utilisation.

\begin{itemize}
    \item \textbf{Avantages :}
    \begin{itemize}
        \item Très bonne précision (inférieure à 1 \%)
        \item Principe simple et largement utilisé
        \item Méthode non-intrusive, sans contact avec le gaz
    \end{itemize}
    
    \item \textbf{Inconvénients :}
    \begin{itemize}
        \item Nécessite l'installation permanente d'une balance sous la bouteille
        \item Solution encombrante et peu pratique pour un usage domestique
        \item Sensible aux déplacements ou chocs de la bouteille
        \item Coût relativement élevé pour une balance précise et fiable
        \item Obligation de calibrer le système avec le poids exact de la bouteille vide
    \end{itemize}
\end{itemize}

\begin{figure}[H]
    \centering
    \includegraphics[width=0.6\textwidth]{mesure_pesage.png}
    \caption{Exemple de solution de mesure du niveau de gaz par pesage}
    \label{fig:solution_pesage}
\end{figure}

\subsubsection{Mesure par pression}

\textbf{Principe :}  
La mesure par pression repose sur la mesure de la pression hydrostatique exercée par le liquide au fond du récipient. Dans un réservoir ouvert, cette pression est directement proportionnelle à la hauteur de liquide. Cependant, dans le cas d'une bouteille de gaz sous pression, cette méthode nécessite un accès direct à l'intérieur du récipient afin d'installer un capteur de pression.

\begin{itemize}
    \item \textbf{Avantages :}
    \begin{itemize}
        \item Bonne précision lorsque la mesure est correctement réalisée
        \item Relation directe entre pression mesurée et niveau de liquide
        \item Peu influencée par la nature du liquide
    \end{itemize}
    
    \item \textbf{Inconvénients :}
    \begin{itemize}
        \item \textbf{Méthode intrusive} nécessitant le perçage de la bouteille
        \item Risque important de fuite de gaz
        \item Installation complexe et dangereuse
        \item \textcolor{red}{\textbf{Méthode non conforme aux exigences de sécurité pour les bouteilles de gaz domestique}}
    \end{itemize}
\end{itemize}

\begin{figure}
    \centering
    \includegraphics[width=0.6\linewidth]{chapters/mesure_pression.png}
    \caption{Solution de mesure du niveau de gaz par mesure de la pression interne}
    \label{mesure_pression}
\end{figure}

\subsubsection{Mesure ultrasonore (Solution retenue)}

\textbf{Principe :}  
La mesure ultrasonore repose sur l'émission d'ondes ultrasonores à travers la paroi métallique de la bouteille. Ces ondes se propagent à l'intérieur de la bouteille et se réfléchissent sur l'interface entre la phase gazeuse et la phase liquide. En mesurant le temps de parcours aller-retour des ondes, il est possible de déterminer avec précision la distance entre le capteur et la surface du liquide, permettant ainsi d'estimer le niveau de gaz restant sans avoir à ouvrir ou modifier la bouteille.

\begin{itemize}
    \item \textbf{Avantages :}
    \begin{itemize}
        \item \textbf{Méthode non-intrusive} ne nécessitant aucune modification de la bouteille
        \item Précision acceptable pour un usage domestique (environ ±5 \%)
        \item Temps de réponse rapide (quelques secondes)
        \item Faible sensibilité aux conditions environnementales
        \item Coût raisonnable des composants électroniques
        \item Technologie éprouvée et largement utilisée en milieu industriel
        \item Faible consommation énergétique compatible avec fonctionnement sur batterie
        \item Installation simple sous la bouteille sans perçage ni modification
    \end{itemize}
    
    \item \textbf{Inconvénients :}
    \begin{itemize}
        \item Nécessite un bon couplage acoustique entre le capteur et la paroi
        \item Calibration initiale indispensable pour chaque type de bouteille
        \item Performances dépendantes de l'état de surface de la bouteille
        \item Sensibilité aux vibrations lors de la mesure
        \item Précision légèrement inférieure à la méthode par pesage
    \end{itemize}
\end{itemize}

\begin{figure}[H]
    \centering
    \includegraphics[width=0.6\textwidth]{chapters/mseure_ultrason.png}
    \caption{Principe de fonctionnement de la détection ultrasonore}
    \label{fig:principe_ultrason}
\end{figure}

\subsection{Tableau comparatif des technologies}

\begin{table}[H]
    \centering
    \caption{Comparaison multicritère des solutions}
    \label{tab:comparaison_solutions}
    \small
    \renewcommand{\arraystretch}{1.3}
    \begin{tabular}{l c c c}
        \toprule
        \textbf{Critère} & \textbf{Pesage} & \textbf{Pression}  & \textbf{Ultrason} \\
        \midrule
        Précision & Excellente & Bonne & Bonne \\
        (Note /5) & 5/5 & 4/5 & 4/5 \\
        \midrule
        Non-intrusif & Oui & Non  & Oui \\
        (Note /5) & 5/5 & 0/5  & 5/5 \\
        \midrule
        Sécurité & Bonne & Risque & Excellente \\
        (Note /5) & 4/5 & 2/5  & 5/5 \\
        \midrule
        Coût & Élevé & Moyen & Moyen \\
        (Note /5) & 2/5 & 3/5 & 4/5 \\
        \midrule
        Facilité & Moyenne & Complexe & Moyenne \\
        (Note /5) & 3/5 & 2/5 & 3/5 \\
        \midrule
        Fiabilité & Bonne & Moyenne & Bonne \\
        (Note /5) & 4/5 & 3/5 & 4/5 \\
        \midrule
        \textbf{TOTAL} & \textbf{23/30} & \textbf{14/30} & \textbf{25/30} \\
        \bottomrule
    \end{tabular}
\end{table}

\subsection{Justification du choix de la solution ultrasonore}

Au vu de l'analyse comparative, la technologie ultrasonore s'impose comme le meilleur compromis pour notre application :

\begin{enumerate}
    \item \textbf{Conformité aux contraintes de sécurité} : Aucune modification de la bouteille, absence de risque de fuite, composants basse tension uniquement.
    
    \item \textbf{Précision suffisante} : L'objectif de ± 5\% est atteignable avec une calibration appropriée et un traitement du signal optimisé.
    
    \item \textbf{Coût maîtrisé} : Les capteurs ultrasonores sont disponibles à des prix raisonnables (10 000 - 15 000 FCFA pièce selon la qualité).
    
    \item \textbf{Facilité d'implémentation} : La technologie est bien documentée avec de nombreuses ressources disponibles pour Arduino et microcontrôleurs embarqués.
    
    \item \textbf{Faible consommation} : Compatible avec l'objectif d'autonomie de plusieurs semaines sur batterie grâce aux modes de veille profonde.
    
    \item \textbf{Fiabilité prouvée} : Cette technologie est utilisée avec succès dans de nombreuses applications industrielles (mesure de niveau dans réservoirs, contrôle non destructif, détection d'objets).
    
    \item \textbf{Installation simple} : Fixation rapide sous la bouteille sans outillage spécialisé ni compétences techniques particulières.
\end{enumerate}

\section{Analyse Fonctionnelle Globale}
\label{sec:analyse_fonctionnelle}

\subsection{Modèle général du système}

Le système de détection de gaz repose sur une architecture distribuée comprenant deux dispositifs matériels indépendants communiquant via technologie LoRa longue portée :

\begin{figure}[H]
    \centering
    \includegraphics[width=0.4\textwidth]{modele.png}
    \caption{Modèle général du système}
    \label{fig:modele_general}
\end{figure}

\begin{enumerate}
    \item \textbf{L'utilisateur} : Personne utilisant la bouteille de gaz et souhaitant connaître le niveau restant sans avoir à se déplacer jusqu'à la bouteille.
    
    \item \textbf{Le dispositif de mesure} : Module autonome fixé sous la bouteille, effectuant périodiquement les mesures ultrasonores et transmettant les données via LoRa. Fonctionne sur batterie Li-Po rechargeable.
    
    \item \textbf{Le dispositif d'affichage} : Boîtier mural équipé d'un écran LCD, de LEDs et d'un buzzer, recevant les données du capteur et affichant le niveau de gaz en temps réel. Déclenche des alarmes visuelles et sonores en cas de niveau critique.
\end{enumerate}

\subsection{Diagramme de contexte}

Le diagramme de contexte illustre les interactions entre le système et son environnement :

\begin{figure}[H]
    \centering
    \includegraphics[width=0.8\textwidth]{diagramme_contexte.png}
    \caption{Diagramme de contexte du système}
    \label{fig:diagramme_contexte}
\end{figure}

Le système est principalement composé de deux sous-ensembles : \textbf{le dispositif de mesure} et \textbf{le dispositif d'affichage}.

Le dispositif de mesure, fixé sous la bouteille de gaz, est chargé de collecter les données physiques liées au niveau de GPL à l'aide du capteur ultrasonique DYP-L06. Ces données sont traitées localement par un microcontrôleur Arduino Nano, puis transmises sans fil au dispositif d'affichage via communication LoRa 433 MHz.

Le dispositif d'affichage, installé dans un endroit accessible de l'habitation (mur, étagère, plan de travail), sert d'interface visuelle pour l'utilisateur. Il affiche le niveau de gaz en pourcentage sur un écran LCD I2C, utilise des LEDs pour indication rapide de l'état, et déclenche une alarme sonore via buzzer en cas de niveau critique. L'utilisateur peut interagir avec le système via des boutons physiques (allumage et arrêt d'alarme).

L'utilisateur interagit principalement avec le dispositif d'affichage pour la consultation locale, tandis que le dispositif de mesure fonctionne de manière totalement autonome avec alimentation sur batterie et gestion intelligente de la consommation énergétique.

\section{Analyse des Besoins}
\label{sec:analyse_besoins}

\subsection{Besoins fonctionnels}

Les besoins fonctionnels décrivent ce que le système doit faire. Ils sont classés par priorité selon la méthode MoSCoW :

\begin{table}[H]
    \centering
    \caption{Besoins fonctionnels(BF) classés par priorité}
    \label{tab:besoins_fonctionnels}
    \renewcommand{\arraystretch}{1.3}
    \begin{tabular}{p{2cm}p{4.5cm}p{9cm}}
        \toprule
        \textbf{Priorité} & \textbf{Besoin} & \textbf{Description} \\
        \midrule
        \multicolumn{3}{c}{\textbf{MUST HAVE (Indispensable)}} \\
        \midrule
        BF01 & Mesure du niveau & Le système doit mesurer le niveau de gaz avec une précision de $\pm$5\% via capteur ultrasonique \\
        BF02 & Affichage local & Le niveau doit être affiché en temps réel sur un écran LCD I2C avec bargraphe \\
        BF03 & Alerte niveau bas & Une alerte visuelle (LED) et sonore (buzzer) doit être déclenchée quand le niveau < 20\% \\
        BF04 & Autonomie capteur & Le dispositif de mesure doit fonctionner au moins 30 jours sur batterie Li-Po \\
        BF05 & Communication LoRa & Transmission fiable des données sur plusieurs kilomètres avec portée minimale de 500m en urbain \\
        \midrule
        \multicolumn{3}{c}{\textbf{SHOULD HAVE (Important)}} \\
        \midrule
        BF06 & Indicateurs LED & Affichage visuel rapide de l'état sur le dispositif d'affichage (vert/orange/rouge) \\
        BF07 & Alarme sonore & Buzzer actif paramétrable pour alertes critiques \\
        BF08 & Boutons physiques & Interaction utilisateur (allumage écran, arrêt alarme) \\
        \midrule
        \multicolumn{3}{c}{\textbf{COULD HAVE (Souhaitable)}} \\
        \midrule
        BF09 & Historique local & Mémorisation des dernières mesures sur l'afficheur \\
        BF10 & Application de suivi & Utilisation d'une application locale (mobile ou web) pour le suivi à distance du niveau de gaz \\
        \midrule
        \multicolumn{3}{c}{\textbf{WON'T HAVE (Non prioritaire)}} \\
        \midrule
        BF11 & Détection de fuites & Alerte en cas de fuite détectée (capteur MQ-6 additionnel) \\
        BF12 & Géolocalisation & Localisation des points de vente de bouteilles \\
        BF13 & Commande vocale & Intégration assistants vocaux (Alexa, Google Home) \\
        \bottomrule
    \end{tabular}
\end{table}

\subsection{Besoins non fonctionnels}

Les besoins non fonctionnels définissent les contraintes de qualité du système :

\subsubsection{Performance}
\begin{itemize}
    \item \textbf{Temps de réponse} : Mesure complète et transmission en moins de 10 secondes
    \item \textbf{Précision} : Erreur maximale de $\pm$5\% sur l'ensemble de la plage de mesure
    \item \textbf{Répétabilité} : Écart-type inférieur à 2\% sur 20 mesures consécutives
    \item \textbf{Latence LoRa} : Délai maximal de 5 secondes entre mesure et affichage
    \item \textbf{Portée LoRa} : Communication fiable jusqu'à 2-5 km en zone dégagée, 500m-1km en milieu urbain
    \item \textbf{Fréquence de mesure} : Actualisation toutes les 15 minutes pour optimiser l'autonomie
\end{itemize}

\subsubsection{Fiabilité}
\begin{itemize}
    \item \textbf{Disponibilité} : Taux de disponibilité supérieur à 99\% pour le système de mesure
    \item \textbf{Durée de vie} : Minimum 5 ans en utilisation normale
    \item \textbf{MTBF} : Temps moyen entre pannes supérieur à 10 000 heures
    \item \textbf{Robustesse thermique} : Résistance aux variations de température (-10°C à +50°C)
    \item \textbf{Taux de perte LoRa} : Taux de paquets perdus inférieur à 1\% en conditions normales
    \item \textbf{Récupération} : Reconnexion automatique en cas de perte temporaire de communication
\end{itemize}

\subsubsection{Utilisabilité}
\begin{itemize}
    \item \textbf{Installation capteur} : Fixation sous bouteille en moins de 3 minutes sans outils
    \item \textbf{Installation affichage} : Montage mural ou sur support en moins de 5 minutes
    \item \textbf{Appairage LoRa} : Connexion automatique au démarrage sans configuration manuelle
    \item \textbf{Apprentissage} : Utilisation intuitive sans formation préalable
    \item \textbf{Lisibilité} : Affichage LCD visible à 3 mètres de distance
\end{itemize}

\subsubsection{Sécurité}
\begin{itemize}
    \item \textbf{Électrique} : Utilisation exclusive de composants basse tension ($\leq$ 5V)
    \item \textbf{Isolation} : Aucun contact électrique avec la bouteille métallique
    \item \textbf{Protection batterie} : Circuit TP4056 avec protection surcharge/décharge/court-circuit
    \item \textbf{Boîtier étanche} : Protection IP54 minimum pour le dispositif de mesure
    \item \textbf{Certifications} : Conformité aux normes IEC 60335 (appareils domestiques)
    \item \textbf{RF} : Respect de la réglementation bande ISM 433 MHz (puissance max 10 mW ERP)
\end{itemize}

\subsubsection{Maintenabilité}
\begin{itemize}
    \item \textbf{Modularité} : Architecture permettant le remplacement de composants défaillants
    \item \textbf{Diagnostic} : LEDs d'état sur chaque dispositif pour identification rapide des problèmes
    \item \textbf{Batterie} : Remplacement facile des batteries sans démontage complet
    \item \textbf{Mises à jour} : Possibilité de reprogrammation du firmware via port UART
    \item \textbf{Documentation} : Manuel technique détaillé avec schémas électroniques et netlists
    \item \textbf{Calibration} : Procédure de recalibration accessible sans équipement spécialisé
\end{itemize}

\subsubsection{Énergie}
\begin{itemize}
    \item \textbf{Autonomie mesure} : Minimum 30 jours sur batterie Li-Po 1500 mAh en mode normal
    \item \textbf{Autonomie affichage} : Minimum 12-16 heures sur batteries 18650 en utilisation continue
    \item \textbf{Modes de veille} : Implémentation de deep sleep pour Arduino et LoRa entre mesures
    \item \textbf{Indication batterie} : Affichage du niveau de batterie sur LCD et LED dédiée
    \item \textbf{Recharge} : Temps de recharge complet inférieur à 4 heures (capteur) via USB
\end{itemize}

\subsection{Scénarios d'utilisation principaux}

\subsubsection{Scénario 1 : Installation initiale}

\begin{enumerate}
    \item L'utilisateur fixe le dispositif de mesure sous la bouteille de gaz à l'aide de la sangle ajustable ou du système de fixation adhésif.
    \item Il connecte la batterie Li-Po au module TP4056 et la place dans le boîtier du capteur.
    \item Il installe le dispositif d'affichage à l'endroit souhaité (cuisine, salon) et insère la batterie plate Li-Po.
    \item Il allume les deux dispositifs via leurs interrupteurs respectifs.
    \item Le système effectue une auto-vérification : le capteur clignote une LED verte, l'écran LCD affiche un message de démarrage.
    \item L'appairage LoRa se fait automatiquement (fréquence 433 MHz préprogrammée).
    \item La procédure de calibration se fait automatiquement sur le dispositif d'affichage.
    \item Le capteur ultrasonique effectue plusieurs mesures pour détecter le niveau actuel et ajuste les paramètres.
    \item La mesure initiale est effectuée et transmise via LoRa, puis affichée sur le LCD en pourcentage.
\end{enumerate}

\subsubsection{Scénario 2 : Consultation quotidienne}

\begin{enumerate}
    \item L'utilisateur se trouve dans la pièce où est installé le dispositif d'affichage.
    \item Il appuie sur le bouton d'allumage pour activer le rétroéclairage du LCD (si éteint).
    \item Le niveau de gaz actuel s'affiche immédiatement en pourcentage (ex: 67\%).
    \item Les LEDs indiquent l'état : verte (>50\%), orange (20-50\%), rouge (<20\%).
    \item Le système repasse automatiquement en veille d'affichage après 30 secondes sans interaction.
\end{enumerate}

\subsubsection{Scénario 3 : Alerte de niveau bas}

\begin{enumerate}
    \item Le dispositif de mesure détecte lors d'une mesure périodique que le niveau est descendu sous le seuil de 20\%.
    \item Il transmet immédiatement cette information critique via LoRa au dispositif d'affichage.
    \item Le buzzer(actif) du dispositif d'affichage émet un signal sonore intermittent.
    \item La LED rouge clignote de manière continue.
    \item L'écran LCD affiche "NIVEAU BAS - 20\%" avec une icône d'avertissement.
    \item L'utilisateur peut désactiver temporairement l'alarme sonore en appuyant sur le bouton "SILENCE", mais la LED rouge continue de clignoter.
    \item L'alarme se réactivera automatiquement après 36 heures si le niveau n'a pas augmenté.
\end{enumerate}

\section{Contraintes du projet}

\subsubsection{Contraintes techniques}

\begin{table}[H]
    \centering
    \caption{Contraintes techniques}
    \label{tab:contraintes_techniques}
    \renewcommand{\arraystretch}{1.3}
    \begin{tabular}{p{5cm}p{10cm}}
        \toprule
        \textbf{Contrainte} & \textbf{Description} \\
        \midrule
        Non-intrusive & Aucune modification de la bouteille autorisée \\
        Compatibilité & Doit fonctionner avec bouteilles 6kg, 12,5kg, 35kg \\
        Paroi métallique & Le capteur ultrasonique doit traverser l'acier (épaisseur 2-3mm) \\
        Température & Fonctionnement de -10°C à +50°C \\
        Humidité & Résistance à 10-90\% d'humidité relative \\
        Vibrations & Résistance aux vibrations domestiques courantes \\
        Alimentation & Fonctionnement sur batterie rechargeable sans alimentation secteur \\
        Communication & Portée LoRa minimum 500m en milieu urbain, 2km en zone dégagée \\
        Fréquence RF & Utilisation bande ISM 433 MHz (libre de licence) \\
        Encombrement & Dispositif mesure: diamètre max 90mm, hauteur max 30mm \\
        Étanchéité & Protection IP54 minimum pour le dispositif de mesure \\
        \bottomrule
    \end{tabular}
\end{table}

\subsubsection{Contraintes économiques}

\begin{itemize}
    \item \textbf{Budget prototype global} : Maximum 250 000 FCFA pour système complet
    \item \textbf{Répartition budgétaire} :
    \begin{itemize}
        \item Dispositif de mesure : 50 000 - 100 000 FCFA
        \item Dispositif d'affichage : 50 000 - 100 000 FCFA
        \item PCB et boîtiers : 10 000 - 30 000 FCFA
    \end{itemize}
    \item \textbf{Composants} : Priorité aux composants disponibles localement (Cameroun) ou via fournisseurs internationaux (AliExpress, Mouser)
    \item \textbf{Coût cible production} : Coût unitaire de production série inférieur à 250 000 FCFA
    \item \textbf{Maintenance} : Coût de maintenance annuel inférieur à 5 000 FCFA (remplacement batteries)
\end{itemize}

\subsubsection{Contraintes temporelles}

\begin{table}[H]
    \centering
    \caption{Planning prévisionnel détaillé}
    \label{tab:planning_detaille}
    \renewcommand{\arraystretch}{1.3}
    \begin{tabular}{c p{6cm} c p{6cm}}
        \toprule
        \textbf{Phase} & \textbf{Activités} & \textbf{Durée} & \textbf{Livrables} \\
        \midrule
        1 & Étude préliminaire & 2 semaines & Rapport d'étude, choix LoRa \\
        2 & Conception détaillée & 4 semaines & Schémas électroniques, netlists \\
        3 & Approvisionnement & 2 semaines & Composants achetés \\
        4 & Réalisation matérielle & 5 semaines & Prototypes fonctionnels (3 dispositifs) \\
        5 & Développement firmware & 4 semaines & Code Arduino \\
        6 & Tests communication LoRa & 1 semaine & Validation portée et fiabilité \\
        7 & Calibration capteur & 2 semaines & Procédure validée \\
        9 & Tests et validation & 2 semaines & Résultats des tests \\
        10 & Documentation & 2 semaines & Rapport final complet \\
        \bottomrule
    \end{tabular}
\end{table}

\begin{center}
    \textbf{TOTAL :} \textbf{25 semaines}
\end{center}

\chapter{Conception du Système Global}
\label{chap:conception_globale}

Suite à l'analyse des besoins et des contraintes, ce chapitre détaille la conception technique du système proposé. Il présente l'architecture globale, justifie les choix technologiques opérés et décrit les stratégies mises en œuvre pour garantir la fiabilité, l'autonomie et la sécurité du dispositif.

\section{Architecture Générale du Système}
\label{sec:architecture_generale}

\subsection{Vue d'ensemble}

Pour répondre aux contraintes d'éloignement et d'autonomie, le système repose sur une architecture distribuée. Nous avons opté pour la technologie radio LoRa, qui assure une liaison robuste entre l'extérieur (lieu de stockage du gaz) et l'intérieur de l'habitation.

Le système s'articule autour de trois entités distinctes :

\begin{figure}[H]
    \centering
    \includegraphics[width=0.6\textwidth]{chapters/archi_cloud.png}
    \caption{Architecture distribuée du système LoRa}
    \label{fig:architecture_lora}
\end{figure}

\begin{itemize}
    \item \textbf{Le dispositif de mesure (Capteur)} : Placé directement sous la bouteille, ce module autonome a pour unique fonction de mesurer le niveau de gaz et de transmettre l'information. Il est conçu pour être "invisible" à l'usage et très économe en énergie.
    
    \item \textbf{Le dispositif d'affichage (Récepteur)} : Situé dans la pièce de vie, il joue le rôle d'interface homme-machine. Il reçoit les données et informe l'utilisateur via un écran et des signaux lumineux ou sonores, sans nécessiter de connexion internet.
    
    \item \textbf{La passerelle LoRa (Optionnelle)} : Cet élément permet d'ouvrir le système vers l'Internet des Objets (IoT). Elle fait le pont entre le réseau local LoRa et le Cloud, permettant ainsi l'usage de l'application mobile.
\end{itemize}

Cette approche modulaire offre une grande flexibilité : le système est parfaitement fonctionnel en mode "local" (capteur + afficheur) pour les utilisateurs sans internet, tout en étant extensible vers une solution connectée complète.

\subsection{Description des dispositifs matériels}

\begin{figure}[H]
    \centering
    \includegraphics[width=0.8\textwidth]{chapters/module.png}
    \caption{Schéma fonctionnel des deux dispositifs}
    \label{fig:modules_dispositif}
\end{figure}


\subsubsection{Le dispositif de mesure (Module Capteur)}
Ce module est le cœur du système d'acquisition. Installé sous la bouteille, il doit faire face à des contraintes d'encombrement et d'autonomie sévères.

Sa conception s'articule autour du capteur ultrasonore \textbf{DYP-L06}, choisi pour sa capacité à mesurer à travers le métal et sa faible consommation. Le pilotage est assuré par un \textbf{Arduino Nano}, qui orchestre les mesures et la mise en veille. La transmission est confiée au module \textbf{LoRa SX1278}, configuré pour une portée optimale en milieu urbain (jusqu'à 1 km).

L'alimentation est fournie par une batterie Li-Po de 3.7V, gérée par un module de charge TP4056 qui assure la sécurité électrique. L'ensemble est intégré dans un boîtier compact (Ø90mm) et étanche (IP54), conçu pour se fixer discrètement sous la bouteille.

\textbf{Stratégie de fonctionnement :}
Pour maximiser l'autonomie, le dispositif ne fonctionne pas en continu. Il suit un cycle strict : \\
1.  \textbf{Réveil} périodique (ex: toutes les 15 minutes). \\
2.  \textbf{Mesure} ultrasonore et calcul du niveau. \\
3.  \textbf{Transmission} de la donnée via LoRa. \\
4.  \textbf{Mise en veille profonde} immédiate de tous les composants. \\

Cette stratégie permet de réduire drastiquement la consommation moyenne, offrant une autonomie théorique de plusieurs semaines.

\subsubsection{Le dispositif d'affichage (Module Récepteur)}
Ce module est conçu pour être installé confortablement dans l'habitation. Il centralise les informations et gère les alertes.

Il partage la même base technologique (Arduino Nano et LoRa SX1278) mais fonctionne en réception. L'interface utilisateur est composée d'un \textbf{écran LCD I2C} pour l'affichage précis et de \textbf{LEDs tricolores} pour une lecture rapide de l'état (Vert/Orange/Rouge). Un buzzer est intégré pour les alertes critiques.

L'alimentation est ici plus robuste, basée sur deux batteries 18650 en série, régulées à 5V, permettant une utilisation prolongée sans recharge fréquente.

Le système gère trois états :
\begin{itemize}
    \item \textbf{Veille} : Écran éteint, radio en écoute (consommation réduite).
    \item \textbf{Actif} : Écran allumé sur demande utilisateur ou réception de données.
    \item \textbf{Alerte} : Activation des signaux sonores et visuels en cas de niveau critique.
\end{itemize}

\subsubsection{La passerelle LoRa (Module Gateway - Optionnel)}
Pour les utilisateurs souhaitant une connectivité étendue, la passerelle joue le rôle de pont. Basée sur un microcontrôleur \textbf{ESP32} (choisi pour sa connectivité Wi-Fi native), elle écoute les messages LoRa et les retransmet vers un serveur distant.

Contrairement aux autres modules, elle est alimentée sur secteur, ce qui lui permet d'être en écoute permanente et de gérer simultanément plusieurs bouteilles si nécessaire.

\subsection{Architecture de l'application mobile}

L'application mobile constitue le terminal de supervision avancé. Elle ne communique pas directement avec les capteurs, mais interroge le serveur Cloud (Backend).

Son rôle dépasse le simple affichage : elle transforme les données brutes en informations utiles. Elle permet notamment de visualiser l'historique de consommation, de recevoir des notifications en cas de niveau bas où que l'on soit, et d'utiliser des algorithmes prédictifs pour estimer la date de remplacement de la bouteille.


\subsection{Architecture du backend}

Le Backend, développé avec le framework \textbf{Spring Boot}, est le cerveau de la partie connectée. Il assure la persistance des données et l'exécution de la logique métier complexe.

\begin{figure}[H]
    \centering
    \includegraphics[width=0.6\textwidth]{chapters/archi_spring.png}
    \caption{Architecture du backend Spring Boot}
    \label{fig:architecture_backend}
\end{figure}

L'architecture logicielle suit le modèle MVC (Modèle-Vue-Contrôleur) pour garantir une séparation claire des responsabilités :

\begin{enumerate}
    \item \textbf{Les Contrôleurs REST} gèrent les entrées/sorties. Ils reçoivent les mesures de la passerelle et répondent aux requêtes de l'application mobile, tout en sécurisant les accès via des jetons JWT.
    \item \textbf{Les Services Métier} contiennent l'intelligence du système : validation des données, détection d'anomalies, calculs statistiques et déclenchement des notifications.
    \item \textbf{La Couche de Données (DAO)} interagit avec la base de données PostgreSQL pour stocker de manière pérenne les utilisateurs, les bouteilles et l'historique des mesures.
\end{enumerate}



\section{Choix Technologiques}
\label{sec:choix_technologiques}

\subsection{Choix du microcontrôleur : Arduino Nano}

Le choix du microcontrôleur est critique pour l'équilibre entre performance, taille et consommation. Nous avons sélectionné l'\textbf{Arduino Nano} pour équiper à la fois le capteur et l'afficheur.

\subsubsection{Justification du choix}

Plusieurs arguments justifient cette décision :

\begin{itemize}
    \item \textbf{Compacité} : Avec ses dimensions réduites (18×45mm), il s'intègre parfaitement dans l'espace restreint sous la bouteille.
    \item \textbf{Efficacité énergétique} : Il supporte des modes de veille profonde ("Deep Sleep") essentiels pour atteindre l'autonomie visée.
    \item \textbf{Standardisation} : L'utilisation du même composant sur les deux modules simplifie la maintenance, le développement du code et la gestion des stocks.
    \item \textbf{Accessibilité} : Son faible coût et sa grande disponibilité en font un choix pertinent pour un déploiement à grande échelle.
\end{itemize}



\subsubsection{Caractéristiques techniques détaillées}

\begin{table}[H]
    \centering
    \caption{Spécifications techniques Arduino Nano}
    \label{tab:specs_arduino}
    \begin{tabular}{ll}
        \toprule
        \textbf{Paramètre} & \textbf{Valeur} \\
        \midrule
        Microcontrôleur & ATmega328P (8-bit AVR) \\
        Fréquence horloge & 16 MHz \\
        Mémoire Flash & 32 KB (dont 2 KB bootloader) \\
        SRAM & 2 KB \\
        EEPROM & 1 KB \\
        Tension fonctionnement & 5V (VIN accepte 7-12V) \\
        Tension logique & 5V (compatible 3.3V via diviseur) \\
        Dimensions & 18 × 45 mm \\
        Poids & ~7g \\
        Pins I/O digitaux & 14 (dont 6 PWM) \\
        Pins analogiques & 8 (ADC 10-bit) \\
        Consommation active & ~20 mA \\
        Consommation veille profonde & <5 mA (avec LowPower lib) \\
        Interfaces & UART, I2C, SPI \\
        Connecteur USB & Mini-USB (programmation + serial monitor) \\
        \bottomrule
    \end{tabular}
\end{table}



\subsection{Choix du capteurs ultrasonores}

\subsubsection{Analyse et justification du choix}
La contrainte majeure du projet est de mesurer le niveau de gaz sans modifier la bouteille (pour des raisons de sécurité et de réglementation). La technologie ultrasonore s'est imposée comme la seule solution viable permettant une mesure "à travers" la paroi métallique.

Nous avons sélectionné le capteur \textbf{DYP-L06}. Contrairement aux capteurs standards (type HC-SR04), ce modèle est étanche (IP67) et conçu pour être couplé à une surface solide. Il offre une précision de $\pm$1\%, largement suffisante pour notre besoin, et fonctionne sur une plage de tension compatible avec notre batterie.

\subsubsection{Principe de fonctionnement}
Le principe physique repose sur la mesure du temps de vol (Time of Flight). Le capteur, positionné au sommet de la bouteille, émet une onde ultrasonore qui traverse le gaz.

Lorsque cette onde rencontre la surface du liquide (GPL), elle est réfléchie vers le capteur. En mesurant le temps aller-retour $\Delta t$, et connaissant la vitesse du son dans le gaz $v_{gaz}$, on déduit la distance $d$ entre le haut de la bouteille et le liquide :

\[d = \frac{v_{gaz} \times \Delta t}{2}\]

Le niveau de remplissage est ensuite obtenu par soustraction par rapport à la hauteur totale de la bouteille.

\[\text{Niveau (\%)} = \frac{H_{tot} - d}{H_{tot}} \times 100\]


\subsubsection{Configuration mécanique du capteur}

\begin{figure}[H]
    \centering
    \includegraphics[width=0.6\textwidth]{chapters/schema_capteur.png}
    \caption{Positionnement du capteur ultrasonore sur la bouteille}
    \label{fig:position_capteur}
\end{figure}

L'installation du capteur est non-destructive. Il est maintenu fermement contre la paroi métallique (généralement sous la bouteille ou sur le dessus selon le modèle) à l'aide d'un adhésif industriel ou d'une sangle magnétique.

Un point crucial est l'application d'un **gel de couplage** (ou pâte thermique) entre le capteur et le métal. Ce gel élimine la fine couche d'air qui empêcherait la transmission des ultrasons, garantissant ainsi la fiabilité de la mesure.

\subsubsection{Avantages de la solution retenue}
Cette approche présente un compromis optimal :
\begin{itemize}
    \item \textbf{Sécurité} : Aucun contact avec le gaz, aucun perçage.
    \item \textbf{Simplicité} : Un seul capteur suffit, réduisant le coût et la complexité.
    \item \textbf{Efficacité} : La mesure est directe et peu influencée par la composition chimique exacte du mélange propane/butane.
\end{itemize}

\subsection{Choix de l'écran : LCD I2C}

Pour l'interface utilisateur, nous avons privilégié la lisibilité et la simplicité. L'écran LCD I2C a été retenu au détriment de l'OLED.

\subsubsection{Justification du choix LCD I2C}
Bien que moins moderne que l'OLED, le LCD offre une meilleure lisibilité à distance (essentiel pour un affichage mural) et une consommation maîtrisée une fois le rétroéclairage éteint. De plus, son interface I2C simplifie considérablement le câblage (seulement 4 fils), augmentant la fiabilité globale du montage.



\subsection{Choix du module de communication : LoRa}

L'adoption de la technologie LoRa (Long Range) représente une évolution majeure par rapport aux solutions Bluetooth classiques. Ce choix est motivé par la nécessité de traverser les obstacles (murs, dalles) séparant souvent la cuisine du lieu de stockage du gaz.

\subsubsection{Justification du choix LoRa}
LoRa offre une portée de plusieurs centaines de mètres en milieu urbain et une excellente pénétration des structures en béton, là où le Bluetooth peinerait à dépasser 10 mètres. De plus, sa très faible consommation en veille est parfaitement adaptée à notre usage intermittent (envoi de quelques octets toutes les 15 minutes).

Nous utilisons le module \textbf{SX1278} fonctionnant sur la bande libre 433 MHz. La communication est configurée en mode "Point-à-Point" (P2P), ce qui permet aux modules de dialoguer directement sans dépendre d'une infrastructure réseau externe ou d'un abonnement opérateur.

\subsubsection{Module LoRa choisi : SX1278}

Nous utilisons le chipset \textbf{Semtech SX1276/SX1278} intégré dans modules breakout compatibles Arduino (RA-02, RFM95W, E32 selon disponibilité fournisseurs).

\textbf{Caractéristiques techniques SX1278 :}
\begin{itemize}
    \item Chipset : Semtech SX1276/SX1278 (référence industrie)
    \item Fréquences supportées : 137-1020 MHz (version 433 MHz utilisée)
    \item Puissance émission : Réglable -4 dBm à +20 dBm (max 100mW)
    \item Sensibilité réception : Jusqu'à -148 dBm (mode SF12, BW 125kHz)
    \item Interface : SPI (MISO, MOSI, SCK, NSS, RESET)
    \item Tension alimentation : 3.3V (compatible sortie Arduino Nano)
    \item Consommation TX +20dBm : ~120 mA
    \item Consommation RX : ~12 mA
    \item Consommation veille : <1 µA
    \item Spreading Factor : SF7 à SF12 (compromis débit/portée configurable)
    \item Bandwidth : 125 kHz, 250 kHz, 500 kHz
    \item Débit données : 0.3 kbps (SF12) à 37.5 kbps (SF7)
\end{itemize}

\subsubsection{Configuration retenue pour le projet}

\textbf{Mode de communication : Point-à-Point (P2P)}

Le mode point-à-point direct a été privilégié pour le fonctionnement autonome capteur <-> afficheur :

\begin{itemize}
    \item \textbf{Simplicité} : Pas de join procedure, pas de serveur réseau LoRaWAN nécessaire
    \item \textbf{Rapidité} : Latence minimale émission→réception (<1s)
    \item \textbf{Indépendance} : Système fonctionnel même sans connexion Internet/infrastructure
    \item \textbf{Coût} : Pas d'abonnement opérateur LoRaWAN requis
    \item \textbf{Confidentialité} : Données restent locales (aucun transit réseau externe)
\end{itemize}

La passerelle optionnelle peut recevoir les mêmes trames P2P et les relayer vers le cloud, offrant le meilleur des deux mondes.

\textbf{Paramètres radio configurés :}
\begin{itemize}
    \item \textbf{Fréquence porteuse} : 433.0 MHz (bande ISM)
    \item \textbf{Spreading Factor} : SF10 (compromis portée/débit)
    \item \textbf{Bandwidth} : 125 kHz (standard)
    \item \textbf{Coding Rate} : 4/5 (protection erreurs)
    \item \textbf{Puissance TX} : +17 dBm (50mW, conforme réglementation)
    \item \textbf{Préambule} : 8 symboles
    \item \textbf{CRC} : Activé (détection erreurs transmission)
\end{itemize}

Ces paramètres offrent portée typique 1-2 km en urbain avec débit ~1 kbps, largement suffisant pour payload <50 bytes toutes les 15 minutes.


\subsection{Choix technologiques logiciels}

\subsubsection{Firmware embarqué : Arduino IDE + C/C++}
Le développement embarqué a été réalisé en C/C++ via l'IDE Arduino. Ce choix nous donne accès à des bibliothèques optimisées pour la gestion de l'énergie (LowPower) et la communication radio, accélérant ainsi la phase de prototypage.

\subsubsection{Firmware passerelle : ESP-IDF / Arduino Core ESP32}
Pour la passerelle, nous utilisons les capacités natives de l'ESP32 pour gérer simultanément la pile LoRa et la pile Wi-Fi, assurant le pontage des données vers le Cloud via des requêtes HTTP sécurisées.

\subsubsection{Backend : Spring Boot}
Côté serveur, le choix de Spring Boot (Java) garantit une robustesse industrielle. Il permet de structurer proprement l'API, de gérer la sécurité des données et d'évoluer facilement vers une architecture micro-services si le projet venait à grandir.

\textbf{Fonctions principales du backend :}
\begin{itemize}
    \item \textbf{Réception données passerelle} : Endpoint POST /api/measurements acceptant payloads JSON
    \item \textbf{Gestion entités métier} : CRUD complet users, bottles, measurements, alerts
    \item \textbf{Persistance base de données} : Stockage relationnel PostgreSQL avec transactions ACID
    \item \textbf{Algorithmes prédiction} : Calcul tendances consommation, estimation jours restants (régression linéaire, moyennes mobiles exponentielles)
    \item \textbf{Notifications push} : Intégration Firebase Cloud Messaging pour alertes mobiles
    \item \textbf{API REST documentée} : Endpoints exposés avec Swagger/OpenAPI pour app mobile
    \item \textbf{Authentification/autorisation} : JWT tokens, rôles utilisateurs
    \item \textbf{Monitoring/logs} : Actuator endpoints, logs structurés JSON
\end{itemize}

\textbf{Stack technique utilisée :}
\begin{itemize}
    \item \textbf{Langage} : Java 21 LTS
    \item \textbf{Framework} : Spring Boot 3.5
    \item \textbf{Base de données} : PostgreSQL 16.0
    \item \textbf{ORM} : Spring Data JPA (Hibernate implémentation)
    \item \textbf{Documentation API} : SpringDoc OpenAPI (Swagger UI)
    \item \textbf{Sécurité} : Spring Security + JWT
    \item \textbf{Build} : Maven
    \item \textbf{Déploiement} : Docker
\end{itemize}



\section{Principe de calibration}
\label{sec:calibration}

Pour garantir des mesures fiables malgré la diversité des bouteilles (6kg, 12.5kg, 35kg), une procédure de calibration est indispensable. Elle permet d'adapter le système à la géométrie spécifique de chaque récipient.

Nous avons mis en place une **calibration à deux points**, simple à réaliser pour l'utilisateur :
\begin{enumerate}
    \item \textbf{Calibration à vide (0\%)} : Le système mesure la distance jusqu'au fond de la bouteille ($d_{vide}$).
    \item \textbf{Calibration à plein (100\%)} : Le système mesure la distance jusqu'à la surface du liquide d'une bouteille neuve ($d_{plein}$).
\end{enumerate}

Ces valeurs sont stockées dans la mémoire permanente (EEPROM) de l'Arduino. Par la suite, le niveau de gaz est calculé par interpolation linéaire :

\[\text{Niveau (\%)} = \left(1 - \frac{d_{mesure} - d_{plein}}{d_{vide} - d_{plein}}\right) \times 100\]

Cette méthode permet de s'affranchir des variations de hauteur ou de forme du fond de la bouteille, garantissant une précision constante.


\subsection{Améliorations possibles (évolutions futures)}

\begin{itemize}
    \item \textbf{Calibration multi-points} : 3 points (vide, 50\%, plein) pour correction non-linéarités (fond bombé)
    \item \textbf{Compensation température} : Ajustement vitesse son selon température GPL mesurée (capteur DS18B20)
    \item \textbf{Auto-calibration} : Détection automatique niveaux extrêmes sur usage long terme
    \item \textbf{Profils bouteilles} : Base de données dimensions constructeurs (stockage multi-profils EEPROM)
\end{itemize}

\section{Gestion de l'énergie}
\label{sec:gestion_energie}

L'autonomie est un critère critique pour un objet connecté domestique. Notre stratégie énergétique diffère selon le module : une économie maximale pour le capteur (difficile d'accès) et un compromis performance/autonomie pour l'afficheur.

\subsection{Dispositif de mesure (capteur)}
Le capteur passe 99\% de son temps en "sommeil profond". Il ne se réveille que quelques millisecondes pour effectuer sa mesure et sa transmission.

\textbf{Alimentation :}
\begin{itemize}
    \item Batterie Li-Po 3.7V (1000-2000 mAh selon encombrement disponible)
    \item Module TP4056 avec protections :
    \begin{itemize}
        \item Surcharge : coupure 4.2V
        \item Décharge profonde : coupure 2.5V (protection chimie Li-Po)
        \item Court-circuit : MOSFET FS8205A
    \end{itemize}
    \item Régulateur AMS1117-3.3V (dropout 1.3V) : 3.7V batterie → 3.3V stable LoRa/capteur
    \item Recharge USB 5V (1A max) via connecteur Micro-USB ou USB-C
\end{itemize}

\textbf{Consommation détaillée par phase :}
\begin{table}[H]
    \centering
    \caption{Profil consommation dispositif de mesure}
    \begin{tabular}{lcc}
        \toprule
        \textbf{Phase} & \textbf{Courant} & \textbf{Durée} \\
        \midrule
        Deep sleep (Arduino + LoRa) & <5 mA & 14 min 59 s \\
        Réveil + init & 20 mA & 100 ms \\
        Mesure ultrasonique & 35 mA & 200 ms \\
        Transmission LoRa +17dBm & 120 mA & 400 ms \\
        Retour veille & 10 mA & 100 ms \\
        \midrule
        \textbf{Moyenne sur cycle 15 min} & \multicolumn{2}{c}{\textbf{~8 mA}} \\
        \bottomrule
    \end{tabular}
\end{table}

Grâce à cette gestion fine, la consommation moyenne chute à environ 8 mA. Avec une batterie standard de 1500 mAh, nous estimons l'autonomie réelle à environ **40 jours** avec une mesure tous les quarts d'heure. Cette durée peut être doublée en espaçant les mesures à 30 minutes.

\subsection{Dispositif d'affichage (récepteur)}
L'afficheur consomme davantage car il doit rester à l'écoute des signaux radio. Pour compenser, nous utilisons des batteries de plus grande capacité (18650) et coupons l'écran LCD lorsqu'il n'est pas regardé.

\textbf{Alimentation :}
\begin{itemize}
    \item 2× batteries 18650 Li-ion 3.7V en série = 7.4V nominal (6.0-8.4V selon charge)
    \item Capacité typique : 2×2500 mAh = 2500 mAh (série conserve capacité, double tension)
    \item Porte-batterie avec BMS (Battery Management System) intégré :
    \begin{itemize}
        \item Protection surcharge : coupure 8.4V (4.2V/cellule)
        \item Protection décharge : coupure 6.0V (3.0V/cellule)
        \item Équilibrage cellules (balancing)
    \end{itemize}
    \item Convertisseur DC-DC step-down LM2596 : 7.4V → 5V stable (efficacité 92\%)
    \item Sortie 5V alimente : Arduino Nano VIN, LCD I2C, LED, buzzer
\end{itemize}

\textbf{Consommation par composant :}
\begin{table}[H]
    \centering
    \caption{Consommation dispositif d'affichage}
    \begin{tabular}{lcc}
        \toprule
        \textbf{Composant} & \textbf{Courant (actif)} & \textbf{Courant (veille)} \\
        \midrule
        Arduino Nano & 20 mA & 5 mA (LED power) \\
        LoRa SX1278 RX continu & 12 mA & <1 µA (sleep) \\
        LCD I2C rétroéclairage ON & 25 mA & 0 mA (éteint) \\
        LED tricolore & 15 mA & 0 mA \\
        Buzzer actif & 30 mA & 0 mA \\
        LM2596 quiescent & 5 mA & 5 mA \\
        \midrule
        \textbf{Total veille écran} & \multicolumn{2}{c}{\textbf{22 mA}} \\
        \textbf{Total actif affichage} & \multicolumn{2}{c}{\textbf{62 mA}} \\
        \textbf{Total alerte (LCD+LED+buzzer)} & \multicolumn{2}{c}{\textbf{107 mA}} \\
        \bottomrule
    \end{tabular}
\end{table}

L'autonomie estimée est d'environ **4 à 5 jours** en usage normal. Bien que plus faible que celle du capteur, elle reste acceptable car le remplacement des batteries est aisé dans l'espace de vie.

\subsection{Passerelle LoRa (gateway)}
La passerelle étant alimentée sur secteur, aucune contrainte énergétique ne s'applique. Elle privilégie la performance et la disponibilité pour assurer le lien temps réel avec le Cloud.

\section{Sécurité du système}
\label{sec:securite}

La manipulation de gaz inflammable impose une rigueur absolue en matière de sécurité. Notre conception intègre la sécurité à trois niveaux : électrique, mécanique et fonctionnelle.

\subsection{Sécurité électrique}
L'ensemble du système fonctionne en **Très Basse Tension de Sécurité (TBTS)**, inférieure à 12V, éliminant tout risque d'électrocution. De plus, une isolation galvanique totale est assurée : aucun composant électrique n'est en contact direct avec le métal de la bouteille, le capteur étant isolé par son boîtier plastique et le système de fixation.

Les batteries au lithium, potentiellement instables, sont systématiquement protégées par des circuits dédiés (BMS) contre les surcharges, les courts-circuits et les décharges profondes.

\subsection{Sécurité mécanique et installation}
Le dispositif est conçu pour être **non-intrusif**. Il ne nécessite aucun perçage ni modification de la bouteille, préservant ainsi son intégrité structurelle conformément aux normes en vigueur. La fixation est robuste (adhésif industriel ou sangle) pour éviter tout détachement accidentel, et les boîtiers sont étanches (IP54) pour résister à un usage en extérieur.

\subsection{Sécurité fonctionnelle}
Le logiciel intègre des mécanismes de surveillance (Watchdog) pour détecter les pannes. Si le capteur cesse d'émettre (panne ou batterie vide), l'afficheur signale une erreur "CAPTEUR HORS LIGNE" après un délai de sécurité. Les seuils d'alerte (20\%, 15\%, 5\%) sont définis pour laisser à l'utilisateur le temps de réagir avant la panne sèche.

\subsection{Sécurité données et vie privée}
Enfin, la confidentialité est assurée par le chiffrement des communications vers le Cloud (HTTPS/TLS). En local, la communication LoRa utilise des identifiants anonymes, ne transmettant aucune donnée personnelle, uniquement des niveaux techniques.

\chapter{Conception et Réalisation des Appareils}
\label{chap:appareil}

Ce chapitre se concentre sur la mise en œuvre pratique des deux modules matériels principaux : le dispositif de mesure (émetteur) et le dispositif d'affichage (récepteur). Nous y détaillerons la conception électronique, l'implémentation logicielle embarquée, ainsi que l'intégration mécanique de chaque appareil.

\section{Appareil 1 : L'Émetteur (Mesure)}
\label{sec:emetteur}

L'émetteur est placé sous la bouteille de gaz. Son rôle est de mesurer la distance entre le fond de la bouteille et la surface du gaz liquide, puis de transmettre cette information par LoRa.

\subsection{Composants matériels}

Le choix des composants a été guidé par les contraintes de compacité, d'autonomie et de fiabilité.

\begin{itemize}
    \item \textbf{Microcontrôleur Arduino Nano} : Comme justifié au chapitre \ref{chap:conception_globale}, sa petite taille (18x45mm) et sa faible consommation en mode veille sont idéales pour une intégration dans un boîtier discret. Il orchestre l'ensemble des opérations : réveil, lecture du capteur, traitement des données et transmission.
    
    \begin{figure}[H]
        \centering
        \includegraphics[width=0.8\textwidth]{arduino nano.png}
        \caption{Microcontrôleur Arduino Nano}
        \label{fig:arduino_nano}
    \end{figure}

    \item \textbf{Capteur ultrasonore DYP-L06} : Ce capteur industriel a été préféré aux modèles grand public pour sa capacité à effectuer des mesures à travers le métal et sa sortie série (UART) qui simplifie l'interfaçage. Il fournit directement une mesure de distance, réduisant la charge de traitement sur le microcontrôleur.
    
    \begin{figure}[H]
        \centering
        \includegraphics[width=0.5\textwidth]{capteur_sonore.jpeg}
        \caption{Capteur DYP-L06}
        \label{fig:capteur}
    \end{figure}
    \item \textbf{Module LoRa SX1278 (433MHz)} : Ce module assure la communication longue portée. Il est interfacé avec l'Arduino Nano via le bus SPI. Sa configuration (SF10, 125kHz, +17dBm) a été choisie pour offrir un excellent compromis entre portée, robustesse face aux obstacles et faible consommation, comme détaillé dans la section \ref{sec:choix_technologiques}.
     \begin{figure}[H]
        \centering
        \includegraphics[width=0.5\textwidth]{lora.png}
        \caption{Module Lora avec antenne}
        \label{fig:Lora}
    \end{figure}
    \item \textbf{Circuit d'alimentation} : L'autonomie est assurée par une batterie Li-Po de 3.7V (1500 mAh). Elle est gérée par un module \textbf{TP4056} qui intègre les protections essentielles contre la surcharge et la décharge profonde. Un régulateur de tension LDO (Low-Dropout) abaisse la tension à 3.3V pour alimenter le module LoRa de manière stable.
\end{itemize}


\subsection{Implémentation logicielle}

Le firmware de l'émetteur est optimisé pour une consommation d'énergie minimale. Il fonctionne selon un cycle veille/activité strict.

\subsubsection{Algorithme de fonctionnement}
L'algorithme, illustré par l'organigramme de la figure \ref{fig:organigramme_emetteur}, suit les étapes ci-dessous dans une boucle infinie :
\begin{enumerate}
    \item \textbf{Mise en veille profonde (Deep Sleep)} : Le microcontrôleur et le module LoRa sont placés dans un état de très faible consommation pendant une durée de 15 minutes.
    \item \textbf{Réveil} : Le timer interne du microcontrôleur le réveille.
    \item \textbf{Lecture du capteur} : Le firmware envoie une commande au capteur DYP-L06 via la liaison série et attend la réception de la trame de réponse. Cette trame hexadécimale, commençant par l'octet `0xFF`, contient la mesure de distance sur deux octets. Un checksum est vérifié pour garantir l'intégrité des données.
    \item \textbf{Calcul du niveau} : La distance brute en millimètres est extraite de la trame. Le niveau en pourcentage est ensuite calculé en utilisant la fonction de transfert linéaire issue de la calibration (voir section \ref{sec:calibration}) : 
    \[\text{Niveau (\%)} = \left(1 - \frac{d_{mesure} - d_{plein}}{d_{vide} - d_{plein}}\right) \times 100\]
    \item \textbf{Préparation de la charge utile (Payload)} : Les données (niveau en pourcentage et un timestamp) sont encapsulées dans une structure de données compacte pour minimiser le temps de transmission.
    \item \textbf{Transmission LoRa} : Le module LoRa est activé, la charge utile est envoyée via la fonction `LoRa.beginPacket()`, puis la transmission est effectuée.
    \item \textbf{Retour en veille} : Immédiatement après la transmission, le système retourne en veille profonde pour le cycle suivant.
\end{enumerate}

\begin{lstlisting}[language=C++, caption=Structure des données transmises]
struct SensorData {
  uint8_t level; // Niveau en pourcentage (0-100)
  uint8_t battery; // Niveau de batterie (0-100)
};
\end{lstlisting}


\section{Appareil 2 : Le Récepteur (Affichage et Alerte)}
\label{sec:recepteur}

Le récepteur est l'interface homme-machine (IHM) du système. Placé à un endroit visible dans l'habitation (cuisine, salon), il a pour mission de recevoir les données de l'émetteur, d'afficher le niveau de gaz de manière claire et de générer des alertes pertinentes pour l'utilisateur.

\subsection{Composants matériels}

\begin{itemize}
    \item \textbf{Arduino Nano} : Le même microcontrôleur est utilisé pour simplifier la maintenance et le développement. Il gère la réception LoRa, le pilotage de l'écran, des LEDs, du buzzer et la lecture des boutons.
    
    \item \textbf{Module LoRa SX1278 (433MHz)} : Identique à celui de l'émetteur, il est configuré en mode réception continue pour être constamment à l'écoute des paquets de données.
    
    \item \textbf{Écran LCD 16x2 I2C} : L'interface I2C a été choisie pour sa simplicité de câblage (4 fils seulement : VCC, GND, SDA, SCL). L'écran affiche le niveau en pourcentage, un bargraphe visuel et des messages d'état.
    
    \begin{figure}[H]
        \centering
        \includegraphics[width=0.7\textwidth]{LCD_arduino.jpg}
        \caption{Écran LCD 16x2 avec interface I2C}
        \label{fig:lcd_i2c}
    \end{figure}
    \item \textbf{LED RGB} : Une LED tricolore fournit un indicateur d'état rapide et visible à distance, changeant de couleur en fonction du niveau de gaz.
    
    \begin{figure}[H]
        \centering
        \includegraphics[width=0.5\textwidth]{led_rgb.jpeg}
        \caption{LED RGB pour indication visuelle}
        \label{fig:led_rgb}
    \end{figure}
    \item \textbf{Buzzer actif} : Il génère une alerte sonore puissante lorsque le niveau de gaz atteint un seuil critique, attirant l'attention de l'utilisateur même s'il ne regarde pas l'écran.
    
    \begin{figure}[H]
        \centering
        \includegraphics[width=0.5\textwidth]{buzzer_actif.jpeg}
        \caption{Buzzer actif pour signal sonore}
        \label{fig:buzzer_actif}
    \end{figure}

    \item \textbf{Boutons poussoirs} : Deux boutons permettent une interaction utilisateur : un pour allumer/éteindre le rétroéclairage de l'écran afin d'économiser l'énergie, et un autre pour acquitter (silencier) l'alarme sonore.
    
    \item \textbf{Alimentation} : Le récepteur est alimenté par deux batteries 18650 en série, offrant une grande capacité pour une autonomie de plusieurs jours. Un convertisseur DC-DC de type "buck" (LM2596) fournit une tension stable de 5V à l'Arduino et aux périphériques.
    \begin{figure}
        \centering
        \includegraphics[width=0.6\textwidth]{batterie_rechargeable.jpg}
        \caption{Batteries pour alimentation du récepteur}
        \label{fig:batterie}
    \end{figure}
\end{itemize}



\subsection{Implémentation logicielle et logique d'alerte}

Le firmware du récepteur est événementiel : son action principale est de réagir à la réception d'un paquet LoRa ou à l'appui sur un bouton.

\subsubsection{Algorithme de fonctionnement}

\begin{enumerate}
    \item \textbf{Initialisation} : Au démarrage, les périphériques (LCD, LoRa) sont initialisés et un message d'accueil est affiché.
    \item \textbf{Écoute LoRa} : Le programme vérifie en permanence si un paquet LoRa a été reçu via la fonction `LoRa.parsePacket()`.
    \item \textbf{Validation du paquet} : Si un paquet est reçu, le firmware vérifie sa taille pour s'assurer qu'elle correspond à celle de la structure `SensorData`. Le CRC (contrôle de redondance cyclique) activé au niveau du module LoRa garantit l'intégrité des données.
    \item \textbf{Mise à jour des données} : Si le paquet est valide, les variables globales (niveau de gaz, niveau de batterie) sont mises à jour. Un timestamp de la dernière réception est enregistré.
    \item \textbf{Gestion de l'affichage et des alertes} : Une fonction dédiée est appelée pour rafraîchir l'écran LCD et piloter la LED RGB et le buzzer selon la logique suivante :
    \begin{itemize}
        \item \textbf{Niveau > 50\%} : LED Verte, pas d'alarme. Message "Niveau OK".
        \item \textbf{20\% < Niveau < 50\%} : LED Orange (R=255, G=165, B=0), pas d'alarme. Message "Niveau Moyen".
        \item \textbf{Niveau < 20\%} : LED Rouge clignotante et déclenchement du buzzer. Message "NIVEAU BAS".
    \end{itemize}
    \item \textbf{Gestion des boutons} : Le programme scanne l'état des boutons pour allumer/éteindre le rétroéclairage ou arrêter le buzzer.
    \item \textbf{Gestion du timeout} : Une sécurité est implémentée. Si aucun paquet n'est reçu pendant une période définie (ex: 35 minutes, soit deux cycles de mesure manqués), le système considère le capteur comme déconnecté. L'écran affiche "Pas de signal !", et la LED passe au bleu pour signaler une perte de communication.
\end{enumerate}

\begin{figure}[H]
    \centering
    \includegraphics[width=0.8\textwidth]{schema_electrique.png}
    \caption{Circuit électrique des appareils emetteur et récepteur}
    \label{fig:schema_electrique}
\end{figure}

\section{Intégration Mécanique}
\label{sec:integration_mecanique}

L'intégration mécanique vise à protéger les circuits électroniques tout en garantissant une installation simple et une bonne expérience utilisateur. Les boîtiers ont été modélisés en 3D et imprimés en PLA.

\subsection{Boîtier et installation de l'émetteur}

Le boîtier de l'émetteur est conçu pour être compact (diamètre 90mm, hauteur 25mm) et robuste. Il est certifié IP54, le protégeant des poussières et des projections d'eau, ce qui est essentiel pour un usage en extérieur ou en cuisine.

L'installation est un point critique pour la qualité de la mesure. Le boîtier doit être fermement plaqué contre le fond plat de la bouteille. Pour cela, une fine couche de gel de couplage (similaire au gel pour échographie) est appliquée sur la face du capteur afin d'éliminer toute bulle d'air et d'assurer une transmission parfaite des ultrasons. Le maintien est assuré par de puissants aimants néodyme intégrés au boîtier, permettant une fixation et un retrait aisés. L'alignement vertical du capteur est crucial pour que l'écho ultrasonore se réfléchisse correctement sur la surface plane du gaz liquide.


\subsection{Boîtier du récepteur}

Le boîtier du récepteur est conçu pour être à la fois esthétique et fonctionnel. Il peut être posé sur une surface plane ou fixé au mur. Des ouvertures sont prévues pour garantir la visibilité de l'écran LCD et de la LED RGB, ainsi qu'un accès facile aux boutons de commande. Des grilles de ventilation assurent une dissipation thermique adéquate et permettent au son du buzzer de se propager efficacement.

\begin{figure}[H]
    \centering
    \includegraphics[width=0.8\textwidth]{Prototype boitier.png}
    \caption{Prototype des boitiers du récepteur et de l'emetteur}
    \label{fig:boitier_recepteur}
\end{figure}
% \chapter{Développement de l'Application Web}
\label{chap:application}

Ce chapitre présente la conception et le développement de l'application web GasTrack, réalisée avec la bibliothèque React.JS.

\section{Modèle de données}
\label{sec:modele_donnee}

\begin{figure}[H]
    \centering
    \includegraphics[width=0.9\textwidth]{chapters/diagramme_classe.png}
    \caption{Diagramme de classe}
    \label{fig:diagramme_classe}
\end{figure}

Ce modèle permet de structurer les informations échangées entre le backend cloud, l’application web et l’utilisateur.

\subsection{Vue d’ensemble du modèle}

Le modèle de données repose sur quatre classes principales :
\textit{Mesure}, \textit{Bouteille}, \textit{Passerelle} et \textit{Utilisateur}.  
Chaque classe représente une entité clé du système et encapsule les données ainsi que les comportements associés.

\subsection{Description des classes}

\subsubsection{Classe \texttt{Bouteille}}

La classe \texttt{Bouteille} représente la bouteille de gaz surveillée par le dispositif.

\textbf{Rôle :}
Elle centralise les informations statiques et dynamiques relatives à la bouteille, notamment son type, sa capacité et son état de remplissage.

\textbf{Responsabilités principales :}
\begin{itemize}
    \item Stocker les caractéristiques de la bouteille (capacité, type de gaz)
    \item Associer les mesures de niveau effectuées
    \item Fournir l’état courant du niveau de gaz
\end{itemize}

\subsubsection{Classe \texttt{Mesure}}

La classe \texttt{Mesure} modélise une mesure de niveau de gaz réalisée par le capteur ultrasonore.

\textbf{Rôle :}
Elle permet d’enregistrer les résultats des mesures et d’assurer l’historisation des données.

\textbf{Responsabilités principales :}
\begin{itemize}
    \item Stocker la valeur mesurée (niveau de gaz)
    \item Enregistrer la date et l’heure de la mesure
    \item Fournir les données nécessaires aux calculs de consommation et de prédiction
\end{itemize}

Chaque objet \texttt{Mesure} est associé à une unique \texttt{Bouteille}.


\subsubsection{Classe \texttt{Utilisateur}}

La classe \texttt{Utilisateur} représente l’utilisateur final de l’application web.

\textbf{Rôle :}
Elle permet de personnaliser l’utilisation du système et de relier les données à un utilisateur donné.

\textbf{Responsabilités principales :}
\begin{itemize}
    \item Consulter le niveau de gaz en temps réel
    \item Accéder à l’historique de consommation
    \item Visualiser les prédictions de consommation
\end{itemize}

Un utilisateur peut être associé à une ou plusieurs bouteilles.

\subsection{Relations entre les classes}

Les relations entre les classes assurent la cohérence du modèle :

\begin{itemize}
    \item Une \texttt{Bouteille} est associée à plusieurs \texttt{Mesure} (relation un-à-plusieurs), ce qui permet l’historisation des niveaux de gaz.
    \item L’\texttt{Utilisateur} consulte les données de sa \texttt{Bouteille} qui sont mises à jour via la \texttt{Passerelle} connectée au cloud.
\end{itemize}


\section{Analyse Fonctionnelle de l'Application}
\label{sec:analyse_app}

\subsection{Cas d'utilisation}

\begin{figure}[H]
    \centering
    \includegraphics[width=0.9\textwidth]{chapters/usecase.png}
    \caption{Diagramme de cas d'utilisation}
    \label{fig:use_cases_app}
\end{figure}

\subsection{Synthèse des cas d’utilisation}

Le tableau suivant présente une synthèse des principaux cas d’utilisation de l’application web.


\begin{table}[H]
\centering
\caption{Cas d'utilisation de l'application web}
\label{tab:cas_utilisation_app}
\small
\renewcommand{\arraystretch}{1.3}
\begin{tabular}{|p{2.8cm}|p{2.5cm}|p{4cm}|p{3.5cm}|p{2.2cm}|}
\hline
\rowcolor{blue!20}
\textbf{Cas d'utilisation} & 
\textbf{Précon-ditions} & 
\textbf{Scénario nominal} & 
\textbf{Scénarios alternatifs} & 
\textbf{Post-conditions} \\
\hline

\textbf{UC1:} Consulter le niveau de gaz & 
Connexion Internet active & 
\begin{enumerate}[leftmargin=*, nosep]
    \item L'utilisateur accède au site web
    \item Authentification automatique
    \item Récupération des dernières données serveur
    \item Affichage en temps réel
\end{enumerate} & 
\textbf{A1:} Pas d'internet → Mode hors ligne (cache)\newline
\textbf{A2:} Serveur inaccessible → Message d'erreur & 
Niveau de gaz affiché avec précision \\
\hline

\textbf{UC2:} Consulter l'historique & 
Application connectée & 
\begin{enumerate}[leftmargin=*, nosep]
    \item Accès à l'onglet "Historique"
    \item Téléchargement des données archivées
    \item Affichage graphique (jour/semaine/mois)
\end{enumerate} & 
\textbf{A1:} Historique vide → Message "Aucune donnée" & 
Historique visualisé sous forme de courbes \\
\hline

\textbf{UC3:} Obtenir une prédiction & 
Données suffisantes sur le serveur & 
\begin{enumerate}[leftmargin=*, nosep]
    \item Requête de prédiction au backend
    \item Réception du résultat calculé
    \item Affichage de la prédiction
\end{enumerate} & 
\textbf{A1:} Données insuffisantes → "Besoin de plus de données"\newline
\textbf{A2:} Consommation irrégulière → Prédiction imprécise & 
Prédiction affichée avec niveau de confiance \\
\hline

\end{tabular}
\end{table}


\section{Parcours Utilisateur}
\label{sec:parcours_utilisateur}

Cette section décrit le parcours utilisateur au sein de l’application web \textit{GasTrack}.  
Le parcours est basé sur les diagrammes d’activités correspondant aux trois principaux cas d’utilisation : 
la consultation du niveau de gaz, la consultation de l’historique de consommation et l’affichage des prédictions de consommation.

\subsection{Principes généraux du parcours}

Quel que soit le cas d’utilisation, le parcours utilisateur repose sur une séquence commune d’actions :
\begin{itemize}
    \item l'identification sécurisée,
    \item la sélection de la bouteille dans le parc,
    \item l’accès à la fonctionnalité demandée.
\end{itemize}

Cette approche garantit une expérience utilisateur cohérente, intuitive et homogène à travers l’ensemble de l’application.

\subsection{Parcours : Consultation du niveau de gaz}

Le premier parcours permet à l’utilisateur de consulter en temps réel le niveau de gaz restant dans la bouteille.

\begin{enumerate}
    \item L’utilisateur se connecte à l'application web.
    \item L'écran d'accueil affiche la liste des bouteilles associées avec leur dernier niveau connu (remonté par la passerelle).
    \item L’utilisateur sélectionne une bouteille pour voir les détails.
\end{enumerate}

Ce parcours se termine lorsque le niveau de gaz est correctement affiché à l’utilisateur.

\subsection{Parcours : Consultation de l’historique de consommation}

Ce parcours permet à l’utilisateur d’analyser l’évolution de sa consommation de gaz dans le temps.

\begin{enumerate}
    \item Depuis l'écran de détail d'une bouteille, l'utilisateur clique sur l'onglet "Historique".
    \item L’application interroge l'API REST pour obtenir les données agrégées.
\end{enumerate}

L’historique est présenté sous forme chronologique afin de faciliter l’analyse des tendances de consommation.

\subsection{Parcours : Affichage de la prédiction de consommation}

Ce parcours vise à fournir à l’utilisateur une estimation de la durée restante avant épuisement du gaz.

\begin{enumerate}
    \item Le backend calcule périodiquement les prédictions basées sur les nouvelles données LoRa.
    \item L'application affiche cette estimation (ex: "Il vous reste environ 12 jours") directement sur le tableau de bord.
\end{enumerate}

Ce parcours permet d’anticiper le remplacement de la bouteille de gaz et d’améliorer la gestion énergétique.

\subsection{Synthèse du parcours utilisateur}

L'utilisation d'une architecture cloud simplifie grandement le parcours utilisateur en supprimant les étapes techniques de connexion locale et d'appairage à chaque utilisation. L'information est disponible immédiatement après connexion au site.

\begin{figure}[H]
    \centering
    \includegraphics[width=0.9\textwidth]{chapters/activite_historique.png}
    \caption{Diagramme d'activité "Consulter Historique"}
    \label{fig:diagramme_activite}
\end{figure}

\begin{figure}[H]
    \centering
    \includegraphics[width=0.9\textwidth]{chapters/activite_niveau_gaz.png}
    \caption{Diagramme d'activité "Consulter Niveau de Gaz}
    \label{fig:activité_niveau_gaz}
\end{figure}

\begin{figure}[H]
    \centering
    \includegraphics[width=0.9\textwidth]{chapters/activite_prediction.png}
    \caption{Diagramme d'activité "Afficher Prédiction"}
    \label{fig:activité_prevision}
\end{figure}

\section{Interfaces Utilisateur}
\label{sec:interfaces_ui}

\begin{figure}[H]
    \centering
    \includegraphics[width=0.5\textwidth]{appli/Capture d’écran du 2025-12-30 19-37-47.png}
    \caption{Page d'accueil}
    \label{fig:accueil}
\end{figure}

\begin{figure}[H]
    \centering
    \includegraphics[width=0.5\textwidth]{appli/Capture d’écran du 2025-12-30 19-40-01.png}
    \caption{Formulaire d'utilisation 1}
    \label{fig:formulaire_1}
\end{figure}

\begin{figure}[H]
    \centering
    \includegraphics[width=0.5\textwidth]{appli/Capture d’écran du 2025-12-30 19-40-13.png}
    \caption{Formulaire d'utilisation 2}
    \label{fig:formulaire_2}
\end{figure}

\begin{figure}[H]
    \centering
    \includegraphics[width=0.5\textwidth]{appli/Capture d’écran du 2025-12-30 19-40-24.png}
    \caption{Formulaire d'utilisation 3}
    \label{fig:formulaire_3}
\end{figure}

\begin{figure}[H]
    \centering
    \includegraphics[width=0.5\textwidth]{appli/Capture d’écran du 2025-12-30 19-49-56.png}
    \caption{Tableau de bord}
    \label{fig:dashboard}
\end{figure}

\begin{figure}[H]
    \centering
    \includegraphics[width=0.5\textwidth]{appli/Capture d’écran du 2025-12-30 19-50-10.png}
    \caption{Historique}
    \label{fig:historique}
\end{figure}

\begin{figure}[H]
    \centering
    \includegraphics[width=0.5\textwidth]{appli/Capture d’écran du 2025-12-30 19-50-21.png}
    \caption{Prédictions}
    \label{fig:predictions}
\end{figure}

\begin{figure}[H]
    \centering
    \includegraphics[width=0.5\textwidth]{appli/Capture d’écran du 2025-12-30 19-50-33.png}
    \caption{Paramètres}
    \label{fig:params}
\end{figure} % Chapitre supprimé (devenu perspective)
\chapter{Tests et Validation}
\label{chap:tests}

La validation expérimentale constitue une étape cruciale du projet. Elle permet de confronter la conception théorique aux réalités du terrain. Ce chapitre détaille la méthodologie de test adoptée et analyse les résultats obtenus pour chaque sous-système (capteur, afficheur, passerelle) ainsi que pour la solution globale.

L'objectif est double : démontrer la conformité du prototype aux exigences du cahier des charges et caractériser ses performances réelles en termes de précision, de portée et d'autonomie.

\section{Stratégie de Test}

Pour garantir la fiabilité du système, nous avons adopté une approche progressive, dite "bottom-up". Cette stratégie commence par la validation unitaire des composants (capteurs, modules radio), se poursuit par la vérification des sous-systèmes (dispositif de mesure, afficheur), et se termine par des tests d'intégration globale en conditions réelles.

Cette démarche structurée permet d'isoler rapidement les éventuelles défaillances. Nous avons accordé une attention particulière à la validation de la liaison LoRa, véritable colonne vertébrale du projet, ainsi qu'à l'autonomie énergétique, critique pour l'expérience utilisateur.

\section{Tests des Dispositifs Matériels}

Cette première phase vise à qualifier le bon fonctionnement individuel des trois modules matériels avant leur interconnexion.

\subsection{Tests du dispositif de mesure (capteur)}

\subsubsection{Test du capteur ultrasonore DYP-L06}
La capacité du capteur à mesurer le niveau de liquide à travers la paroi métallique est le point le plus critique du système. Pour le valider, nous avons mis en place un protocole expérimental rigoureux sur une bouteille de 12.5 kg.

Le capteur a été fixé avec un gel de couplage acoustique pour optimiser la transmission des ondes. Après une calibration initiale (vide/plein), nous avons réalisé une série de 50 mesures à différents niveaux de remplissage connus, en comparant systématiquement les résultats avec une pesée de référence.

\textbf{Résultats quantitatifs :}
\begin{table}[H]
    \centering
    \caption{Précision mesures ultrasoniques par niveau}
    \begin{tabular}{cccc}
        \toprule
        \textbf{Niveau réel} & \textbf{Niveau mesuré} & \textbf{Écart} & \textbf{Écart-type} \\
        \textbf{(pesée)} & \textbf{(moyen)} & \textbf{absolu} & \textbf{(10 mesures)} \\
        \midrule
        100\% & 99.2\% & -0.8\% & 1.2\% \\
        75\% & 73.5\% & -1.5\% & 1.5\% \\
        50\% & 51.8\% & +1.8\% & 1.8\% \\
        25\% & 26.3\% & +1.3\% & 1.4\% \\
        10\% & 8.7\% & -1.3\% & 2.1\% \\
        0\% & 1.2\% & +1.2\% & 1.6\% \\
        \midrule
        \multicolumn{4}{c}{\textbf{Erreur absolue moyenne : ±1.3\%}} \\
        \multicolumn{4}{c}{\textbf{Écart-type moyen : 1.6\%}} \\
        \bottomrule
    \end{tabular}
\end{table}

\textbf{Analyse des résultats :}
L'analyse des données montre une excellente corrélation entre la mesure ultrasonore et la masse réelle. L'erreur moyenne de 1.3\% est bien inférieure à la tolérance de 5\% fixée initialement. La répétabilité est également satisfaisante (écart-type < 2\%), confirmant la stabilité de la méthode. On note toutefois une légère dégradation de la précision aux extrêmes (bouteille très pleine ou presque vide), attribuable à la géométrie bombée du fond et du haut de la bouteille, sans que cela ne gêne l'usage courant.

\subsubsection{Test du microcontrôleur Arduino Nano et traitement}
Le bon fonctionnement du microcontrôleur a été vérifié sous plusieurs aspects. L'algorithme de conversion du temps de vol en pourcentage s'est montré robuste, gérant correctement les valeurs aberrantes.

Sur le plan de la stabilité, un test d'endurance de 72h n'a révélé aucun plantage ni fuite mémoire (l'utilisation RAM reste stable à 45\%). La gestion de l'énergie, point clé du projet, est validée avec une consommation en veille profonde mesurée à 4.2 mA, conforme aux prévisions.

\subsubsection{Test module LoRa SX1278 émetteur}
Concernant la transmission, nous avons validé la communication SPI entre l'Arduino et le module LoRa. Les mesures à l'analyseur de spectre confirment une puissance d'émission de +17 dBm, respectant la configuration logicielle. Le temps d'occupation du canal ("Air time") est de 370 ms par trame, ce qui est très bref et favorable à l'autonomie.

\subsection{Tests du dispositif d'affichage (récepteur)}

\subsubsection{Test de l'affichage LCD I2C et des alertes}
L'interface utilisateur a fait l'objet d'une attention particulière. Nous avons vérifié la lisibilité de l'écran LCD (confirmée jusqu'à 3.5 mètres) et la réactivité du système.

Le tableau suivant résume la validation des seuils d'alerte :

\begin{table}[H]
    \centering
    \caption{Validation déclenchement alertes par seuil}
    \begin{tabular}{cccc}
        \toprule
        \textbf{Niveau simulé} & \textbf{LED} & \textbf{Buzzer} & \textbf{Message LCD} \\
        \midrule
        75\% & Verte fixe & Inactif & "Niveau: 75\% OK" \\
        45\% & Orange fixe & Inactif & "Niveau: 45\%" \\
        18\% & Rouge clignotante & 3 bips/10s & "ATTENTION BAS"\\
        8\% & Rouge clignotante & Continu & "CRITIQUE 8\%" \\
        \bottomrule
    \end{tabular}
\end{table}

Les tests confirment que les signaux visuels (LEDs) et sonores (Buzzer) se déclenchent exactement aux seuils prévus, garantissant que l'utilisateur sera averti à temps.

\subsubsection{Test module LoRa SX1278 récepteur}
En réception, le module décode correctement 100\% des trames de test. Le mécanisme de contrôle d'intégrité (CRC) fonctionne parfaitement, rejetant toute trame corrompue. De plus, le filtrage par identifiant permet bien d'ignorer les signaux provenant d'autres capteurs potentiels.

\subsubsection{Test autonomie batterie afficheur}
L'autonomie étant un critère de confort, nous avons mesuré la durée de vie réelle des batteries 18650. Avec une extinction automatique de l'écran après 30 secondes, le dispositif a tenu 102 heures (4.25 jours) en fonctionnement continu. Ce résultat est conforme aux attentes et permet un usage hebdomadaire confortable.

\section{Tests de Communication LoRa}

La technologie LoRa étant au cœur de notre architecture pour pallier les limites du Bluetooth, la validation de la portée radio est fondamentale.

\subsection{Test de portée en ligne de vue}
Le premier test s'est déroulé en terrain dégagé (champ libre) pour établir la performance maximale théorique du système.

\textbf{Résultats mesurés :}
\begin{table}[H]
    \centering
    \caption{Portée LoRa en ligne de vue (SF10, BW 125kHz, +17dBm)}
    \begin{tabular}{ccccc}
        \toprule
        \textbf{Distance} & \textbf{Trames} & \textbf{Taux} & \textbf{RSSI} & \textbf{SNR} \\
        \textbf{(m)} & \textbf{reçues/20} & \textbf{réception} & \textbf{(dBm)} & \textbf{(dB)} \\
        \midrule
        100 & 20/20 & 100\% & -52 & +12 \\
        500 & 20/20 & 100\% & -78 & +8 \\
        1000 & 20/20 & 100\% & -95 & +5 \\
        2000 & 19/20 & 95\% & -115 & +1 \\
        3000 & 18/20 & 90\% & -125 & -2 \\
        4000 & 14/20 & 70\% & -132 & -5 \\
        5000 & 8/20 & 40\% & -138 & -8 \\
        \bottomrule
    \end{tabular}
\end{table}

Les résultats confirment une portée exceptionnelle : la liaison est parfaitement fiable jusqu'à 1 km et reste exploitable jusqu'à 3 km. Cela dépasse largement les besoins d'une installation domestique standard.

\subsection{Test de portée en milieu urbain avec obstacles}
Pour évaluer la performance en conditions réelles, nous avons testé le système dans un environnement résidentiel dense, caractérisé par des murs en béton et des obstacles multiples.

\textbf{Résultats mesurés :}
\begin{table}[H]
    \centering
    \caption{Portée LoRa en milieu urbain}
    \begin{tabular}{p{5cm}ccc}
        \toprule
        \textbf{Configuration} & \textbf{Distance} & \textbf{Taux} & \textbf{RSSI} \\
        & \textbf{(m)} & \textbf{réception} & \textbf{(dBm)} \\
        \midrule
        Même pièce & 5 & 100\% & -35 \\
        Pièces adjacentes (1 mur) & 10 & 100\% & -58 \\
        RDC → 1er étage (dalle béton) & 15 & 100\% & -72 \\
        RDC → 2e étage (2 dalles) & 20 & 98\% & -88 \\
        RDC → 3e étage (3 dalles) & 25 & 95\% & -102 \\
        Bâtiments adjacents 50m & 50 & 100\% & -85 \\
        Bâtiments adjacents 100m & 100 & 97\% & -98 \\
        Bâtiments séparés 200m & 200 & 92\% & -112 \\
        Jardin → intérieur 50m (végétation) & 50 & 98\% & -82 \\
        \bottomrule
    \end{tabular}
\end{table}

Ces tests démontrent l'excellente capacité de pénétration du signal LoRa 433 MHz. Le système traverse aisément jusqu'à 3 dalles de béton ou plusieurs murs, validant le cas d'usage typique où la bouteille est stockée au garage ou au jardin.

\subsection{Test de fiabilité et robustesse}
Au-delà de la portée, la robustesse du lien radio a été éprouvée sur 48h continues. Avec un taux de réception de 98.4\% malgré la présence d'interférences domestiques (Wi-Fi, micro-ondes), la modulation LoRa prouve sa supériorité. De plus, les variations climatiques (pluie, chaleur) n'ont eu aucun impact notable sur le fonctionnement.

\section{Tests Logiciels}

La fiabilité du matériel ne serait rien sans un logiciel robuste. Nous avons donc validé chaque couche logicielle.

\subsection{Tests firmware Arduino (capteur + afficheur)}
Les tests logiciels embarqués ont principalement ciblé la gestion de l'énergie et la persistance des données. Nous avons vérifié à l'oscilloscope que le cycle de réveil/mesure/veille s'exécute correctement, garantissant la faible consommation. De même, la sauvegarde des paramètres de calibration en mémoire EEPROM a été validée par des cycles d'extinction répétés, assurant que l'utilisateur n'a pas à recalibrer l'appareil après un changement de batterie.

\section{Tests d'Intégration Système Complet}

Cette phase finale valide la synergie entre tous les composants du projet.

\subsection{Test autonomie batterie en conditions réelles}
Sur la durée, l'autonomie s'est révélée conforme aux calculs théoriques. Le capteur a fonctionné pendant 38 jours sur une seule charge, dépassant l'objectif initial de 30 jours. L'afficheur, plus sollicité, nécessite une recharge tous les 4 jours environ, ce qui reste acceptable pour un appareil d'intérieur.

\subsection{Validation globale du système}
En synthèse, la campagne de tests valide la réussite technique du projet. Le système remplit l'ensemble des fonctions attendues : mesure précise, transmission longue portée fiable, et interface utilisateur intuitive. Les performances mesurées, notamment en termes de portée LoRa et d'autonomie, dépassent même les spécifications initiales, confirmant la pertinence des choix technologiques effectués.

\section{Déploiement et Documentation}

\subsection{Documentation livrée}
Pour assurer la pérennité du projet et faciliter sa reproduction ou son amélioration future, une documentation exhaustive a été produite. Elle comprend les schémas électroniques détaillés, les nomenclatures (BOM), ainsi que les manuels d'installation et d'utilisation.

\subsection{Code source et réutilisabilité}
L'ensemble du code source (Firmware) a été structuré et commenté pour favoriser sa réutilisabilité. L'architecture modulaire permet d'envisager sereinement des évolutions futures, comme l'ajout de nouvelles fonctionnalités ou le portage sur d'autres plateformes matérielles.

\section{Gallerie des tests}
\begin{figure}[H]
    \centering
    \includegraphics[width=0.8\textwidth]{succes_niveau_normal.jpg}
    \caption{Cas du niveau de gaz élevé}
    \label{fig:succes}
\end{figure}

\begin{figure}[H]
    \centering
    \includegraphics[width=0.8\textwidth]{succes_niveau_moyen.jpg}
    \caption{Cas du niveau de gaz moyen}
    \label{fig:moyen}
\end{figure}

\begin{figure}[H]
    \centering
    \includegraphics[width=0.8\textwidth]{succes_niveau_bas.jpg}
    \caption{Cas du niveau de gaz bas}
    \label{fig:bas}
\end{figure}

\begin{figure}[H]
    \centering
    \includegraphics[width=0.8\textwidth]{fonctionnement_buzzer.jpg}
    \caption{Fonctionnement du buzzer}
    \label{fig:buzzer}
\end{figure}
\chapter{Difficultés Rencontrées et Solutions}
\label{chap:difficultes}

Ce chapitre présente les principales difficultés rencontrées tout au long de la réalisation du projet, ainsi que les solutions mises en œuvre pour les surmonter.  
Il met également en évidence les leçons apprises, tant sur le plan technique qu’organisationnel.

\section{Difficultés Matérielles}

\subsection{Difficulté d’acquisition du matériel}

L’une des principales contraintes rencontrées concerne l’acquisition du matériel nécessaire à la réalisation du dispositif embarqué.  
En particulier, la commande du capteur ultrasonore DYP-L06 a connu un retard important, s’étalant sur plus d’un mois. Ce retard a fortement impacté le planning initial du projet.

\textbf{Causes identifiées :}
\begin{itemize}
    \item Disponibilité limitée du capteur sur le marché local.
    \item Délais de livraison prolongés pour les commandes en ligne.
    \item Contraintes logistiques liées à l’importation du matériel.
\end{itemize}

\textbf{Solutions mises en œuvre :}
\begin{itemize}
    \item Réorganisation du planning en priorisant les tâches logicielles.
    \item Étude théorique approfondie du capteur en attendant sa réception.
    \item Simulation et préparation du code embarqué sans matériel physique.
\end{itemize}

\subsection{Contraintes financières}

La mobilisation des fonds nécessaires à l’achat des composants électroniques a également constitué une difficulté majeure.

\textbf{Problèmes rencontrés :}
\begin{itemize}
    \item Budget limité pour un projet de groupe.
    \item Retard dans la contribution financière de certains membres.
\end{itemize}

\textbf{Solutions adoptées :}
\begin{itemize}
    \item Réduction des coûts en choisissant des composants alternatifs lorsque possible.
    \item Mutualisation des ressources entre les membres du groupe.
    \item Planification progressive des achats selon les priorités.
\end{itemize}

\subsection{Disponibilité des bouteilles de gaz pour les tests}

La réalisation des tests en conditions réelles nécessitait la disponibilité de bouteilles de gaz vides et pleines, ce qui n’a pas toujours été évident.

\textbf{Solutions :}
\begin{itemize}
    \item Collaboration avec des particuliers et des commerces locaux.
    \item Utilisation de bouteilles partiellement remplies pour certains tests intermédiaires.
\end{itemize}

\section{Leçons Apprises}

La réalisation de ce projet a permis à l’équipe d’acquérir de nombreuses compétences et enseignements.

\textbf{Principales leçons retenues :}
\begin{itemize}
    \item L’importance d’une bonne planification et d’une gestion réaliste des délais.
    \item La nécessité d’anticiper les problèmes d’approvisionnement matériel.
    \item La valeur du travail en équipe et du partage des connaissances.
    \item L’intérêt de tester progressivement chaque composant avant l’intégration globale.
    \item L’adaptabilité face aux imprévus techniques et organisationnels.
\end{itemize}

Ces expériences ont contribué à renforcer les compétences techniques et organisationnelles des membres de l’équipe et constituent un apport significatif pour leurs futurs projets académiques et professionnels.

\chapter{Limites et Perspectives d'Évolution}
\label{chap:perspectives}

Ce chapitre présente les principales limites du système développé dans le cadre de ce projet,
ainsi que les perspectives d’amélioration et d’évolution possibles.
Il met en évidence les axes de progression tant sur le plan matériel que logiciel, et ouvre la voie
à une éventuelle transformation du prototype en une solution commerciale.

\section{Limites Actuelles}

Malgré les résultats satisfaisants obtenus, le système présente certaines limites inhérentes
au cadre académique et aux ressources disponibles.

\subsection{Limites matérielles}

\begin{itemize}
    \item Le dispositif repose sur un seul capteur ultrasonore, ce qui peut limiter la précision
    dans certaines conditions (inclinaison de la bouteille, vibrations).
    \item La fixation du capteur dépend fortement de la qualité du couplage acoustique,
    pouvant influencer la fiabilité des mesures.
    \item L’autonomie, bien que satisfaisante (environ 30 jours), reste limitée par l’utilisation
    de composants peu optimisés énergétiquement.
    \item Le prototype n’intègre pas encore un boîtier industriel certifié pour un usage prolongé.
\end{itemize}

\subsection{Limites logicielles}

\begin{itemize}
    \item Les algorithmes de prédiction reposent sur des modèles simples et ne tiennent pas
    encore compte de tous les paramètres d’usage (habitudes, saisonnalité).
\end{itemize}

\section{Améliorations Matérielles}

Plusieurs améliorations matérielles peuvent être envisagées afin d’augmenter les performances
et la fiabilité du système.

\begin{itemize}
    \item Utilisation de plusieurs capteurs ultrasonores pour améliorer la précision et introduire
    une redondance des mesures.
    \item Remplacement de l’Arduino Uno par un microcontrôleur plus performant et économe
    en énergie (ESP32, STM32).
    \item Intégration d’un boîtier robuste et étanche, conforme aux normes industrielles.
    \item Ajout de capteurs complémentaires (température, inclinaison) pour corriger et affiner
    les mesures.
    \item Optimisation du circuit d’alimentation pour augmenter l’autonomie de la batterie.
\end{itemize}

\section{Améliorations Logicielles}

Des évolutions logicielles significatives peuvent également être envisagées.

\begin{itemize}
    \item \textbf{Développement d'une application web et mobile :} Création d'une interface connectée pour permettre la consultation des niveaux à distance via Internet.
    \item Mise en place d'une infrastructure backend pour l'historisation des données dans le cloud.
    \item Amélioration des algorithmes de prédiction à l’aide de techniques d’apprentissage
    automatique basées sur l’historique de consommation.
    \item Ajout de notifications intelligentes sur smartphone (alertes personnalisées, prévisions avancées).
\end{itemize}

\section{Évolution vers une Solution Commerciale}

À plus long terme, le projet peut évoluer vers une solution commercialisable.

\subsection{Étapes envisagées}

\begin{itemize}
    \item Validation du prototype à travers des tests intensifs en conditions réelles.
    \item Miniaturisation du dispositif et optimisation du design industriel.
    \item Certification du produit selon les normes de sécurité et de compatibilité électromagnétique.
    \item Déploiement de l'application web sur une infrastructure cloud robuste et scalable.
    \item Mise en place d’une infrastructure backend scalable.
\end{itemize}

\subsection{Vision à long terme}

À terme, la solution pourrait être étendue à :
\begin{itemize}
    \item La gestion intelligente de la consommation énergétique domestique.
    \item L’intégration dans des systèmes domotiques et des plateformes IoT.
    \item Une utilisation à l’échelle industrielle ou commerciale (restaurants, hôtels, entreprises).
\end{itemize}

Ces perspectives montrent que le projet constitue une base solide pour des développements futurs,
tant académiques que professionnels.

\chapter{Conclusion Générale}
\label{chap:conclusion}

\section{Synthèse du Travail Réalisé}

L’objectif principal de ce projet était de concevoir et de réaliser un système intelligent de surveillance du niveau de gaz domestique capable de fonctionner de manière fiable, autonome et sécurisée. Pour répondre à cette problématique réelle, nous avons développé une solution distribuée innovante composée de deux unités complémentaires : un module de mesure autonome fixé sous la bouteille et un module de supervision déporté destiné à l’utilisateur.

L’architecture retenue repose sur la technologie LoRa, choisie pour sa portée étendue, sa robustesse face aux obstacles et sa faible consommation énergétique. Le module capteur, piloté par un microcontrôleur Arduino Nano et associé au capteur ultrasonore DYP-L06, permet une mesure non intrusive du niveau de gaz à travers la paroi métallique. Les données sont ensuite transmises sans fil vers l’unité d’affichage qui assure une visualisation claire et immédiate des informations.

Cette approche modulaire offre une solution à la fois flexible, sécurisée et adaptée aux contraintes domestiques, tout en respectant les exigences techniques du cahier des charges.

\section{Atteinte des Objectifs}

Les objectifs fixés en début de projet ont été atteints de manière satisfaisante :

\begin{itemize}
    \item \textbf{Mesure non intrusive :} L’utilisation des ultrasons permet d’estimer le niveau de gaz sans modification physique de la bouteille, garantissant sécurité et conformité.
    
    \item \textbf{Communication longue portée :} L’intégration de la technologie LoRa a permis de dépasser les limites des communications courte portée et d’assurer une transmission fiable sur plusieurs centaines de mètres.
    
    \item \textbf{Supervision conviviale :} Le module d’affichage avec écran LCD fournit une interface simple et lisible permettant un suivi en temps réel.
    
    \item \textbf{Autonomie énergétique :} L’optimisation de la consommation et l’usage de modes veille assurent un fonctionnement prolongé compatible avec une utilisation quotidienne.
\end{itemize}

Ainsi, la solution développée répond aux exigences fonctionnelles tout en restant réaliste sur les plans technique, économique et opérationnel.

\section{Apports Personnels et Compétences Acquises}

La réalisation de ce projet a constitué une expérience particulièrement enrichissante. Sur le plan technique, elle a permis de consolider des compétences en électronique embarquée, acquisition de données capteurs, communication radio longue portée et intégration matérielle. La confrontation aux contraintes physiques réelles a également permis de mieux comprendre les enjeux de fiabilité, de précision et de robustesse des systèmes embarqués.

Sur le plan organisationnel, le travail en équipe a favorisé le développement de compétences en gestion de projet, répartition des tâches, coordination technique et résolution collaborative de problèmes. Cette démarche nous a sensibilisés aux exigences de rigueur, de validation progressive et de documentation propres aux projets d’ingénierie.

Ce projet représente ainsi une véritable mise en situation professionnelle, proche des conditions réelles de conception d’un système technologique complet.

\section{Ouverture}

Au-delà de ses résultats immédiats, ce travail constitue une base solide pour le développement futur de solutions IoT appliquées à la gestion domestique. Il démontre l’intérêt d’une approche distribuée combinant capteurs intelligents, communication longue portée et interfaces utilisateur adaptées.

Il illustre également le potentiel des technologies embarquées pour répondre à des besoins concrets du quotidien, en améliorant à la fois le confort, la sécurité et l’efficacité énergétique des utilisateurs.

\clearpage
\pagestyle{plain} % Supprime les en-têtes (conserve numéro de page en bas)
\chapter*{Références}
\addcontentsline{toc}{chapter}{Réferences}
\markboth{Bibliographie}{Bibliographie}

\begin{itemize}[label={-}]

\item Arduino. \textit{Arduino Official Documentation}.  
Disponible sur : \url{https://docs.arduino.cc} (consulté en janvier 2026).

\item DYP Sensor. \textit{DYP-L06 Ultrasonic Sensor Datasheet}.  
Disponible sur : \url{https://www.dypcn.com/uploads/DYP-L06.pdf} (consulté en janvier 2026).

\item Buyya, R., \& Dastjerdi, A. V. \textit{Internet of Things: Principles and Paradigms}. Morgan Kaufmann, 2016.

\item Margolis, M. \textit{Arduino Cookbook}. 2\textsuperscript{e} édition, O'Reilly Media, 2016.

\item Valvano, J. W. \textit{Embedded Systems: Real-Time Interfacing to ARM Cortex-M Microcontrollers}. CreateSpace, 2015.

\item Semtech. \textit{LoRa Modulation Basics}.  
Disponible sur : \url{https://www.semtech.com/lora} (consulté en janvier 2026).

\item Kinsler, L. E., Frey, A. R., Coppens, A. B., \& Sanders, J. V.  
\textit{Fundamentals of Acoustics}. John Wiley \& Sons, 2000.

\item Andreas Spiess. \textit{LoRa with Arduino}. YouTube Video.  
Disponible sur : \url{https://www.youtube.com/watch?v=hMOwbNUpDQA} (consulté en janvier 2026).

\item FreeCodeCamp.org. \textit{Internet of Things Full Course}. YouTube.  
Disponible sur : \url{https://www.youtube.com/watch?v=QSIPNhOiMoE} (consulté en janvier 2026).

\item World LPG Association. \textit{Global LPG Market Outlook}.  
Disponible sur : \url{https://www.wlpga.org/publications/global-lpg-market-outlook/} (consulté en janvier 2026).

\item International Energy Agency (IEA). \textit{Energy Access Outlook – Cooking Fuels}.  
Disponible sur : \url{https://www.iea.org/reports/energy-access-outlook-2017} (consulté en janvier 2026).

\end{itemize}
% \endgroup supprimé car on a changé le style de page globalement pour la fin


\end{document}